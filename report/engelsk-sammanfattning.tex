Fixed-wing \acp{uav} are today used in many different areas, from agriculture to search and 
rescue operations. Through various research efforts, they are becoming more and more autonomous. However, the procedure of landing a fixed-wing \ac{uav} 
remains a challenging task, which requires manual input from an experienced pilot. 

This work proposes a novel method which autonomously performs such landings. The main focus is on 
small and light-weight \acp{uav}, for which the wind acts as a major disturbance and has to be taken into account. 
Robustness to other disturbances, such as variations in environmental factors or measurement errors, has also been prioritized during the development 
of this method.

The main contribution of this work consists of a framework in which der\-iva\-tive-free optimization is used to 
calculate a set of waypoints, which are feasible to use in different wind speeds and directions, for a selected \ac{uav} model. 
These waypoints are then combined online using motion planning techniques, to create a trajectory which safely brings the \ac{uav} to a 
position where the landing descent can be initiated. To ensure a safe descent in a predefined area, another non-linear optimization problem is 
formulated and solved. 

Finally, the proposed method is implemented on a real \ac{uav} platform. A number of simulations in different wind conditions are performed, and data from 
a real flight experiment is presented. The results indicate that the method successfully calculates feasible landing sequences in different scenarios, and 
that it is applicable in a real-world landing.


\chapter{Introduction}
\acresetall
\section{Background}
\acp{uav} have many different applications, both in commercial 
usecases such as construction and agriculture, but also in emergency response and personal use. 
\acp{uav} are often primarily divided into two subclasses, multirotors and fixed-wing \acp{uav}. 
While both types of \acp{uav} are becoming more and more autonomous through various research efforts, landing a fixed-wing \ac{uav} remains a challenging task which 
normally requires manual input from an experienced pilot. For small and light-weight \acp{uav}, the presence of wind also acts as a major disturbance which needs to be taken into account when planning 
the landing sequence.
\section{Scope}
This section describes the overall scope of this thesis, the method used as well as where limitations have been made.
\subsection{Problem formulation}
The aim of this thesis is to develop a method for automatically generating feasible 
landing procedures for fixed-wing \acp{uav}, in the presence of wind. The landing sequence should be able to take the \ac{uav} 
from an arbitrary initial position and land safely in a predefined landing area while fulfilling physical constraints of the system. 
This thesis aims to answer the following questions:
\begin{enumerate}
    \item How can sampling-based motion planning techniques be used to generate landing sequences for fixed-wing \acp{uav}?
    \item How can wind effects be taken into account when computing safe landing sequences?
\end{enumerate}

\subsection{Method}
The method proposed in this thesis consists of two main components, the \textit{landing sequence calculation} and the \textit{motion planner}. The goal of the landing sequence calculation is 
to calculate a landing sequence which is used to land the \ac{uav} safely in a predefined landing area $\mathcal{A}$. When the landing sequence is calculated, the goal of the motion planner is to 
calculate a sequence of waypoints $\mathcal{M}$ which, when executed with the tracking controller of the \ac{uav}, takes it from a starting location $x_0$ to a position where the landing sequence can be executed.

Both these components must take the current wind $\windvec$ as well as any obstacles around the landing area into account. The wind is estimated using a wind estimation system onboard the \ac{uav}, while 
obstacles are stored in an obstacle database $\xobst$. Finally, a positioning system onboard the \ac{uav} is used to determine the starting position sent to the motion planner. An overview of the different 
components and their relationship is shown in Figure \ref{fig:sys_overview}.

\begin{figure}[H]
    \begin{center}
        \begin{tikzpicture}[node distance = 3cm, auto]
            \node[block] (init){Landing area input};
            \node[block, right of=init] (land){Landing sequence calculation \\[0.075in] Chapter \ref{cha:landing}};
            \node[block, below of=land] (obst){Obstacle database};
            \node[block, above of=land] (wind){Wind estimation};
            \node[block, right of=land] (mp){Motion planner \\[0.075in] Chapter \ref{cha:motion_planning_fw}};
            \node[block, above of=mp] (gps){Positioning system};
            \node[block, right of=mp] (traj){Waypoint controller};

            \path[line] (init) - > node{$\landing$} (land); 
            \path[line] (obst) - > node[midway, right]{$\xobst$} (land);
            \path[line] (obst) - > node[midway, right]{$\xobst$} (mp);
            \path[line] (wind) - > node[midway, right]{$\windvec$} (land);
            \path[line] (wind) - > node[midway, right]{$\windvec$} (mp);
            \path[line] (land) - > node[midway]{$x_g$} (mp);
            \path[line] (gps) - > node[midway, right]{$x_0$} (mp);
            \path[line] (mp) - > node[midway]{$\mathcal{M}$} (traj);
        \end{tikzpicture}
    \end{center}
    \caption{An overview of the different components in the proposed method.}
    \label{fig:sys_overview}
\end{figure}

\subsection{Limitations}\label{sec:aims}
The components of a general autonomous \ac{uav} are illustrated in Figure \ref{fig:autonomous}. The work in this thesis 
is mainly focused on the motion planning component. However, the tracking controllers and properties of the actual \ac{uav} have to be taken into account 
to ensure feasibility of the generated path. Furthermore, the main focus of this thesis is \acp{uav} using the ArduPilot open source autopilot \cite{arduplane}, 
since the implementation relies on the trajectory controller implemented in ArduPilot.

A large part of this thesis is concerned with the analysis of wind, which is assumed to be constant in space and time. The wind is defined as a vector with magnitude 
$\windspd$ and direction $\winddir$. The wind magnitude $\windspd$ will also be referred to as wind speed and $\winddir$ as wind direction. 

\begin{figure}
    \begin{center}
        \begin{tikzpicture}[node distance = 3cm, auto]
            \node[block, very thin] (init){Behaviour layer};
            \node[block, dashed, below of=init] (user){User};
            \node[block, right of=init, very thick] (mp){Motion planning \\[0.075in] Chapter \ref{cha:motion_planning}};
            \node[block, right of=mp, thick] (ctrl){Tracking controllers \\[0.075in] Chapter \ref{cha:fixed_wing_uav}};
            \node[block, right of=ctrl, thick] (uav){\ac{uav} \\[0.075in] Chapter \ref{cha:fixed_wing_uav}};

            \path[line] (user) -> node[midway, left]{Manual input} (init);
            \path[line] (init) -> node[midway, above, align=left, yshift=.9cm]{High-level \\ commands} (mp);
            \path[line] (mp) -> node[midway, above, align=left, yshift=.9cm]{Reference \\ trajectory} (ctrl);
            \path[line] (ctrl) -> node[midway, above, align=left, yshift=.9cm]{Actuator \\ outputs} (uav);
        \end{tikzpicture}
    \end{center}
    \caption{Components of a general autonomous \ac{uav}. The main focus of this thesis is motion planning, but modeling of the tracking controllers and \ac{uav} is also discussed.}
    \label{fig:autonomous}
\end{figure}

\section{Related work}
The following section presents previous work that is relevant to the subject of this thesis.

\subsection{Motion planning}
Motion planning refers to the task of finding a feasible path between an initial state and a goal state for a given system. 
Since this is an important component of autonomous systems it has 
received increasing research interest during the past decades, with a number of 
different algorithms and methods available. 

\subsubsection{Sampling-based motion planning}
Many motion planning techniques are based on discrete sampling of the continuous state and action space. These methods are either based on 
random sampling - such as in Probabilistic Roadmaps \cite{prm} and Rapidly Exploring Random Trees \cite{rrt}, while others, such as Hybrid $A^*$ \cite{hybrid_astar}, use 
deterministic sampling.

In \cite{2_phase_uav} the $A^*$ algorithm \cite{astar} is used to find kinematically feasible trajectories for fixed-wing \acp{uav} with 
a maximum turn rate while avoiding obstacles. The feasibility of the resulting path is ensured by aligning the dimensions of the 
sampled grid with model parameters of the given \ac{uav}.

In \cite{wind_astar} the results of \cite{optimal_path_trochoidal} are used together with $A^*$ to generate time-optimal trajectories in the 
presence of wind, while also avoiding obstacles. The authors further use the results in \cite{optimal_path_target} to define a modified heuristic function which takes wind into account.

An RRT-based motion planning framework for fixed-wing \acp{uav}, with constraints on both arrival time and final direction 
is proposed in \cite{rrt_uav}.

\subsubsection{Optimal control approach}
The problem of finding time-optimal paths for fixed wing \acp{uav} in uniform winds is often used to formulate an optimal control problem. 
To formulate this problem the following kinematic model is used:
\begin{equation}
    \dot{x}=f(x, u)=
    \begin{bmatrix}
        \airspd\cos\psi + \windspd\cos\winddir\\
        \airspd\sin\psi + \windspd\sin\winddir\\
        u    
    \end{bmatrix} 
\end{equation}
where the input $u$ is the turn-rate. This is generally constrained as per $|u|\leq\dot{\psi}_{\text{max}}$ \cite{optimal_path_target}\cite{optimal_path_trochoidal}.

An important result used in \cite{optimal_path_target} is that earth-fixed goal states become non-stationary in a coordinate frame relative to the air. 
This is used to reformulate the problem as 
finding a path which intersects a virtual target moving from the ground-fix point $x_g$ with the same velocity as the wind but in the opposite direction. 
It is shown that in most cases, the shortest path relative to the earth corresponds to an air-relative shortest path which can be found analytically.
In some cases however, a non-optimal path relative to the air is required to 
intercept the target. A general solution which uses root-finding techniques to cover both cases is also presented.

The approach in \cite{optimal_path_trochoidal} is based on the observation from \cite{course_hdg_wind} that 
constant turn-rate paths in the air-relative frame correspond to \textit{trochoidal} paths in the inertial frame. A trochoid is the path followed by a fix point on a circle which rolls along a line.
They further show that there exists an analytical solution to compute some of the optimal-path candidates, but to find 
all possible optimal paths a non-polynomial equation has to be solved on a two dimensional grid which is computationally expensive.

\subsection{Landing approaches}
The problem of autonomously landing fixed-wing \acp{uav} in different settings has been studied by several authors. 
In many of these works, the problem is defined as landing the \ac{uav} on a runway. A survey of different landing techniques is given in \cite{survey_landing}.
In \cite{emergency_landing} a framework is proposed for emergency landing of fixed-wing \acp{uav} during thrust-loss and uniform wind. The motion planner in this work is based on the 
trochoidal paths discussed in \cite{optimal_path_trochoidal}. The problem of landing fixed-wing \acp{uav} on a moving ground vehicle is studied in, \eg\, \cite{landing_on_vehicle}.

\section{Outline}
Chapter \ref{cha:fixed_wing_uav} introduces general concepts regarding fixed-wing \acp{uav} and wind, as well as the kinematic models and controllers studied in this thesis.
Chapter \ref{cha:motion_planning} gives an overview of motion planning theory. In Chapter \ref{cha:motion_planning_fw}
a motion planning method for fixed-wing \acp{uav} flying in wind is proposed.
Chapter \ref{cha:landing} describes the landing sequence of a fixed-wing \ac{uav} and how landing parameters can be calculated while fulfilling a set of given constraints. 
Chapter \ref{cha:results} presents the implementation of the proposed method on a real \ac{uav} platform, as well as experimental 
results from both simulated and real flight experiments. Finally, these results are discussed and summarized in Chapter \ref{cha:discussion}.
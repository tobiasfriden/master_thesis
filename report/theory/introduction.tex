\chapter{Introduction}

\section{Background}
Unmanned Aerial Vehicles (UAVs) have many different applications, both in commercial usecases such as construction and agriculture, but also in emergency response and personal use. 
UAVs can be divided into two subclasses, multirotors and fixed-wing UAVs. 
While both types of UAVs are becoming more and more autonomous through various research efforts, landing of fixed-wing UAVs remains a challenging task which 
requires manual input from an experienced pilot. For small and light-weight UAVs the presence of wind also acts as a major disturbance which has to be taken into account when planning 
the landing procedure.
\section{Scope}
\subsection{Problem formulation}
The aim of this thesis is to develop a framework for automatically generating feasible 
landing procedures for fixed-wing UAVs, under the presence of winds. The landing procedure should be able to take the UAV 
from an arbitrary initial position and land safely in a predefined landing area while fulfilling physical constraints of the system. 
This thesis aims to answer the following questions:
\begin{enumerate}
    \item How can sampling-based motion planning techniques be used to generate landing sequences for fixed-wing UAVs?
    \item How can safe landings be guaranteed when taking wind effects into account?
\end{enumerate}
\subsection{Aim and delimitations}
The components of a general autonomous system are illustrated in Figure \ref{fig:autonomous}. The work in this thesis 
is mainly focused on the motion planning component. However, the tracking controllers and parameters of the actual UAV have to be taken into account 
to ensure feasibility of the generated path. Furthermore, the main focus of this thesis is UAVs using the ArduPilot open source autopilot \cite{arduplane}.

\begin{figure}
    \begin{center}
        \begin{tikzpicture}[node distance = 3cm, auto]
            \node[block] (init){Behaviour layer};
            \node[block, dashed, below of=init] (user){User};
            \node[block, right of=init] (mp){Motion planning};
            \node[block, right of=mp] (ctrl){Tracking controllers};
            \node[block, right of=ctrl] (uav){UAV};

            \path[line] (user) -> node[midway, left]{Manual input} (init);
            \path[line] (init) -> node[midway, above, align=left, yshift=.9cm]{High-level \\ commands} (mp);
            \path[line] (mp) -> node[midway, above, align=left, yshift=.9cm]{Reference \\ trajectory} (ctrl);
            \path[line] (ctrl) -> node[midway, above, align=left, yshift=.9cm]{Actuator \\ outputs} (uav);
        \end{tikzpicture}
    \end{center}
    \caption{Components of a general autonomous system}
    \label{fig:autonomous}
\end{figure}

\section{Outline}
Chapter \ref{cha:fixed_wing_uav} introduces general concepts regarding fixed-wing UAVs and the kinematic models and controllers studied in this thesis.
Chapter \ref{cha:wind} defines the wind field and discusses methods to estimate wind.
Chapter \ref{cha:motion_planning} gives an overview of necessary motion planning theory. In Chapter \ref{cha:motion_planning_fw} the 
a motion planning framework for fixed-wing UAVs flying in wind is proposed.
Chapter \ref{cha:landing} describes the landing procedure of a fixed-wing UAV and how optimal landing parameters can be calculated. 
In Chapter \ref{cha:implementation} the implementation of the proposed framework on a real UAV platform is discussed. Chapter \ref{cha:results} presents experimental 
results which are discussed in chapter \ref{cha:discussion}.
\section{Related work}
\subsection{Motion planning}
Motion planning refers to the task of finding a feasible path between a initial state and a goal for a given system. Since this is an important component of autonomous systems it has received increasing research interest lately, with an array of different algorithms and methods available. 

\subsubsection{Sampling based motion planning}
Many motion planning techniques are based on discrete sampling of the continuous state - and action space. These methods are either based on 
random sampling, such as in Probabilistic Roadmaps \cite{prm} or Rapidly Exploring Random Trees \cite{rrt} while others such as Hybrid $A^*$ \cite{hybrid_astar} use 
deterministic sampling.

In \cite{2_phase_uav} the $A^*$ algorithm \cite{astar} is used to find kinematically feasible trajectories for fixed-wing UAVs with 
a maximum turn rate while avoiding obstacles. The feasibility of the resulting path is ensured by aligning the dimensions of the 
sampled grid with model parameters of the given UAV.

In \cite{wind_astar} the results in \cite{optimal_path_trochoidal} are used together with $A^*$ to generate time-optimal trajectories in the 
presence of wind, while also avoiding obstacles. They further define the region for where air-relative Dubins paths are not time-optimal 
based on the results in \cite{optimal_path_target} and use these results to modify the heuristic function.

A RRT-based motion planning framework for fixed-wing UAVs, with constraints on both arrival time and final direction is proposed in \cite{rrt_uav}.

\subsubsection{Optimal control approach}
The problem of finding time-optimal paths for fixed wing UAV:s in uniform winds formulated as an optimal control problem has been studied
by different authors. In these works the following kinematic model is used:
\begin{equation}
    \dot{x}=f(x, u)=
    \begin{bmatrix}
        V_a\cos\psi + W\cos\psi_w\\
        V_a\sin\psi + W\sin\psi_w\\
        u    
    \end{bmatrix} 
\end{equation}
where the input is the turn-rate $u$ which is constrained such as $|u|\leq\dot{\psi}_{max}$.
In \cite{optimal_path_target} the problem is reformulated as finding a path which 
intersects a virtual target moving from $x_g$ with the same speed as the wind but in the opposite direction. It is shown that in most cases,
the shortest path is a Dubin's path in the air-relative frame, but in some cases a non-Dubin's path is required to 
intercept the target. The optimal solution is found by defining the function
\begin{equation}
    G(d)=T_a(d) - T_{vt}(d)
\end{equation}
where $d$ is a given distance travelled by the virtual target, $T_a(d)$ is the minimal time at which the 
aircraft reaches the point where the target has travelled this distance and $T_{vt}$ is the required time for the virtual target. 
The optimal solution is found when $G(d)=0$ and can be solved for using numerical root-finding techniques.

The approach in \cite{optimal_path_trochoidal} is based on the observation from \cite{course_hdg_wind} that 
constant turn-rate paths in the air-relative frame correspond to \textit{trochoidal} paths in the inertial frame. 
They further show that there exists an analytical solution to compute some of the optimal-path candidates, but to find 
all possible optimal paths a trandescendal equation has to be solved on a two dimensional grid which is computationally expensive.

\subsection{Landing approaches}
The problem of autonomously landing fixed-wing UAVs in different settings has been studied by several authors. 
In many of these works, the problem is defined as landing the UAV on a runway. A survey of different landing techniques is given in \cite{survey_landing}.
In \cite{emergency_landing} a framework is proposed for emergency landing of fixed-wing UAVs during thrust-lost and uniform wind. The motion planner in this work is based on the 
trochoidal paths discussed in \cite{optimal_path_trochoidal}. The problem of landing fixed-wing UAVs on a moving ground vehicle is studied in \cite{landing_on_vehicle}.
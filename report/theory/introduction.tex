\chapter{Introduction}
\acresetall
\section{Background}
\acp{uav} have many different applications, both in commercial 
usecases such as construction and agriculture, but also in emergency response and personal use. 
\acp{uav} are often primarily divided into two subclasses, multirotors and fixed-wing \acp{uav}. 
While both types of \acp{uav} are becoming more and more autonomous through various research efforts, landing of fixed-wing \acp{uav} remains a challenging task which 
normally requires manual input from an experienced pilot. For small and light-weight \acp{uav}, the presence of wind also acts as a major disturbance which needs to be taken into account when planning 
the landing procedure.
\section{Scope}
This section describes the overall scope of this thesis, as well as where limitations have been made.
\subsection{Problem formulation}
The aim of this thesis is to develop a framework for automatically generating feasible 
landing procedures for fixed-wing \acp{uav}, under the presence of winds. The landing procedure should be able to take the \ac{uav} 
from an arbitrary initial position and land safely in a predefined landing area while fulfilling physical constraints of the system. 
This thesis aims to answer the following questions:
\begin{enumerate}
    \item How can sampling-based motion planning techniques be used to generate landing sequences for fixed-wing \acp{uav}?
    \item How can wind effects be taken into account when computing safe landing sequences?
\end{enumerate}
\subsection{Aim and delimitations}\label{sec:aims}
The components of a general autonomous \ac{uav} are illustrated in Figure \ref{fig:autonomous}. The work in this thesis 
is mainly focused on the motion planning component. However, the tracking controllers and parameters of the actual \ac{uav} have to be taken into account 
to ensure feasibility of the generated path. Furthermore, the main focus of this thesis is \acp{uav} using the ArduPilot open source autopilot \cite{arduplane}.

A large part of this thesis is concerned with the analysis of wind. 
The wind is however assumed to be constant in space and time, and is defined as a constant vector with magnitude 
$\windspd$ and direction $\winddir$. The wind magnitude $\windspd$ will also be referred to as wind speed and $\winddir$ as wind direction. 

\begin{figure}
    \begin{center}
        \begin{tikzpicture}[node distance = 3cm, auto]
            \node[block] (init){Behaviour layer};
            \node[block, dashed, below of=init] (user){User};
            \node[block, right of=init] (mp){Motion planning};
            \node[block, right of=mp] (ctrl){Tracking controllers};
            \node[block, right of=ctrl] (uav){\ac{uav}};

            \path[line] (user) -> node[midway, left]{Manual input} (init);
            \path[line] (init) -> node[midway, above, align=left, yshift=.9cm]{High-level \\ commands} (mp);
            \path[line] (mp) -> node[midway, above, align=left, yshift=.9cm]{Reference \\ trajectory} (ctrl);
            \path[line] (ctrl) -> node[midway, above, align=left, yshift=.9cm]{Actuator \\ outputs} (uav);
        \end{tikzpicture}
    \end{center}
    \caption{Components of a general autonomous \ac{uav}}
    \label{fig:autonomous}
\end{figure}

\section{Outline}
Chapter \ref{cha:fixed_wing_uav} introduces general concepts regarding fixed-wing \acp{uav} and wind, as well as the kinematic models and controllers studied in this thesis.
Chapter \ref{cha:motion_planning} gives an overview of necessary motion planning theory. In Chapter \ref{cha:motion_planning_fw} the 
a motion planning framework for fixed-wing \acp{uav} flying in wind is proposed.
Chapter \ref{cha:landing} describes the landing procedure of a fixed-wing \ac{uav} and how optimal landing parameters can be calculated. 
In Chapter \ref{cha:implementation} the implementation of the proposed framework on a real \ac{uav} platform is discussed. Chapter \ref{cha:results} presents experimental 
results which are discussed in chapter \ref{cha:discussion}.
\section{Related work}
The following section presents previous work that is relevant to the subject of this thesis.
\subsection{Motion planning}
Motion planning refers to the task of finding a feasible path between an initial state and a goal state for a given system. 
Since this is an important component of autonomous systems it has 
received increasing research interest during the past decades, with an array of 
different algorithms and methods available. 

\subsubsection{Sampling based motion planning}
Many motion planning techniques are based on discrete sampling of the continuous state and action space. These methods are either based on 
random sampling, such as in Probabilistic Roadmaps \cite{prm} or Rapidly Exploring Random Trees \cite{rrt}, while others such as Hybrid $A^*$ \cite{hybrid_astar} use 
deterministic sampling.

In \cite{2_phase_uav} the $A^*$ algorithm \cite{astar} is used to find kinematically feasible trajectories for fixed-wing \acp{uav} with 
a maximum turn rate while avoiding obstacles. The feasibility of the resulting path is ensured by aligning the dimensions of the 
sampled grid with model parameters of the given \ac{uav}.

In \cite{wind_astar} the results in \cite{optimal_path_trochoidal} are used together with $A^*$ to generate time-optimal trajectories in the 
presence of wind, while also avoiding obstacles. They further use the results in \cite{optimal_path_target} to define a modified heuristic function which takes wind into account.

A RRT-based motion planning framework for fixed-wing \acp{uav}, with constraints on both arrival time and final direction is proposed in \cite{rrt_uav}.

\subsubsection{Optimal control approach}
The problem of finding time-optimal paths for fixed wing \acp{uav} in uniform winds is often used to formulate an optimal control problem which has been studied
by different authors. In these works the following kinematic model is used:
\begin{equation}
    \dot{x}=f(x, u)=
    \begin{bmatrix}
        \airspd\cos\psi + \windspd\cos\winddir\\
        \airspd\sin\psi + \windspd\sin\winddir\\
        u    
    \end{bmatrix} 
\end{equation}
where the input $u$ is the rate of turn which is constrained such as $|u|\leq\dot{\psi}_{\text{max}}$.

Earth-fixed goal states become non-stationary in a coordinate frame relative to the air. 
This is used in \cite{optimal_path_target} to reformulate the problem as 
finding a path which intersects a virtual target moving from $x_g$ with the same speed as the wind but in the opposite direction. 
It is shown that in most cases, the shortest path relative to the earth corresponds to an air-relative shortest path which can be found analytically.
In some cases however, a non-optimal path relative to the air is required to 
intercept the target. A general solution which uses root-finding techniques to cover both cases is also presented.

The approach in \cite{optimal_path_trochoidal} is based on the observation from \cite{course_hdg_wind} that 
constant rate of turn paths in the air-relative frame correspond to \textit{trochoidal} paths in the inertial frame. A trochoid is the path followed by a fix point on a circle which rolls along a line.
They further show that there exists an analytical solution to compute some of the optimal-path candidates, but to find 
all possible optimal paths a non-polynomial equation has to be solved on a two dimensional grid which is computationally expensive.

\subsection{Landing approaches}
The problem of autonomously landing fixed-wing \acp{uav} in different settings has been studied by several authors. 
In many of these works, the problem is defined as landing the \ac{uav} on a runway. A survey of different landing techniques is given in \cite{survey_landing}.
In \cite{emergency_landing} a framework is proposed for emergency landing of fixed-wing \acp{uav} during thrust-lost and uniform wind. The motion planner in this work is based on the 
trochoidal paths discussed in \cite{optimal_path_trochoidal}. The problem of landing fixed-wing \acp{uav} on a moving ground vehicle is studied in \eg\ \cite{landing_on_vehicle}.
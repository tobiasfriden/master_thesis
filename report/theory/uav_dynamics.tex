\section{UAV dynamics}
A suitable model of the UAV dynamics has to be derived in order to calculate an efficient route. The UAV uses the $L_1$ control law presented in \cite{l1_controller}. This control law navigates from the origin to a goal coordinate by steering towards a reference point $P$ along the line from the origin to the goal which is at distance $L_1$ from the UAV in each time step. Based on this control law, some initial assumptions are made and the UAV state and control vectors are defined.

\begin{assumption}
The UAV will always hold a constant altitude, which means that the route optimization can be performed in two dimensions without taking altitude into account.
\end{assumption}

\begin{assumption}
The UAV is able to hold a constant velocity $V$ relative to the surrounding air. Any internal dynamics related to velocity are neglected.
\end{assumption}

\begin{definition}[UAV state]
The UAV state vector is defined as
\begin{equation}\label{eq:ss_first}
\vec{x}=(\lambda,\phi,\theta)    
\end{equation}
where $(\lambda,\phi)$ is the latitude and longitude coordinate and $\theta$ is the current heading, defined to be zero in the north direction.
\end{definition}

\begin{definition}[UAV control]
The UAV control vector is defined as
\begin{equation}
    \vec{u}=(\lambda_g,\phi_g,V)
\end{equation}
where $(\lambda_g,\phi_g)$ is the latitude and longitude of the desired goal position and $V$ is the desired velocity.
\end{definition}

To derive the state-space model, the following variables are defined:
\begin{equation}\label{eq:ss_variabels}
    \left\{
    \begin{aligned}
        \vec{v}=&V(\sin\theta, \cos\theta)\\
        \vec{r}=&(\lambda, \phi)\\
        \vec{g}=&(\lambda_g, \phi_g)\\
        \theta_g=&\mathrm{atan2}(\lambda_g, \phi_g)\\
        d=&(\vec{r}\times\vec{g}/|\vec{g}|)\cdot\uvec{z}=-\lambda\sin\theta_g + \phi\cos\theta_g 
    \end{aligned}
    \right.
\end{equation}
where $\vec{v}$ is the UAV velocity vector, $\vec{r}$ is the UAV position, $\vec{g}$ is the goal position, $\theta_g$ is the goal heading and $d$ is the cross track error. All the defined variables are shown in Figure~\ref{fig:ss_defs}.

\begin{figure}[htb]
    \begin{center}
    \tikzstyle{point} = [
        circle,
        minimum width=3.5pt,
        inner sep=0,
        fill=black
    ]
    \tikzstyle{my_v} = [
        ->,
        line width = 1.2pt
    ]
    \begin{tikzpicture}[scale=0.85]
        %\draw[help lines](0,0) grid (10,10);
        \coordinate (origin) at (1,1);
        \coordinate (drone) at (2,6);
        \coordinate (goal) at (9,9);
        \coordinate (ref) at (7,7);
        
        \node[anchor=east] at (drone) {$(\lambda, \phi)$};
        \draw[my_v] (drone) -- node[above, near end, anchor=south east]{$\vec{v}$} ++ ({atan(2)}:2.5);
        \draw[dashed] (drone) --++ (45:2);
        \draw[->] ([yshift=0.7cm]drone) arc(90:{atan(2)}:0.7) node[above,midway]{$\theta$};
        \draw[dashed] (drone) --++ (0,1);
        \draw[->] (drone) ++ (0.5,1) arc({atan(2)}:45:{sqrt(1.25)}) node[midway,anchor=223]{$\eta_2$};
        \draw[->] (drone) ++ (45:0.7) arc(45:{atan(0.2)}:0.7) node[midway, anchor=190]{$\eta_1$};
        \draw (drone) -- node[below, anchor=north east]{$d$} ++ (2, -2);
        
        \node[anchor=north west] at (goal) {$(\lambda_g, \phi_g)$};
        
        \draw (drone) -- node[above,sloped]{$L_1$} (ref);
        \node[point] at (ref) {};
        \node[anchor=north west] at (ref) {$P$};
        
        \draw[my_v] (origin) -- node[above, at end]{$\vec{g}$} (goal);
        \draw[my_v] (origin) -- node[left]{$\vec{r}$} (drone);
        \draw[->] (origin) ++ (1,0) arc(0:45:1) node[midway,anchor=west]{$\theta_g$};
        \draw[dashed] (origin) --++ (2,0);
    \end{tikzpicture}
        
    \caption{Definition of UAV State Space variables}
    \label{fig:ss_defs}
    \end{center}
\end{figure}

In the ArduPilot implementation, the distance $L_1$ is calculated as
\begin{equation}\label{eq:ardu_l1}
    L_1=\begin{cases}
        \frac{1}{\pi}\zeta\Delta TV & \mbox{if}\quad |\frac{1}{\pi}\zeta\Delta TV|>|\vec{r}-\vec{g}| \\
        |\vec{r}-\vec{g}| & \mbox{otherwise}
    \end{cases}
\end{equation}
where $\zeta$ is the damping factor and $\Delta T$ is the update period of the controller \cite{arduplane_l1}.

 In each time step, the control law corresponds to following a circular segment with radius
 \begin{equation}
    R=\frac{L_1}{2\sin\eta}
 \end{equation}
 which is tangent to $\vec{v}$ in $(\lambda,\phi)$.
 $\eta$ is defined as the angle between the UAV velocity vector $\vec{v}$ and the line from the UAV to $P$. The centripetal acceleration is then 
\begin{equation}
    a_s=2\frac{V^2}{L_1}\sin\eta
\end{equation}
By introducing a line dividing $\eta$ which is parallel to $\vec{g}$ it can be written as $\eta=\eta_1+\eta_2$ 
where $\eta_1$ is the angle between this line and the line between the UAV 
and $P$, and $\eta_2$ is the angle between $\vec{v}$ and this line.
If the magnitude of $\eta$ is small
\begin{equation}\label{eq:eta_first}
    \sin\eta\approx\eta=\eta_1+\eta_2
\end{equation} 
where
\begin{equation}
    \eta_1\approx \frac{d}{L_1}
\end{equation}
and
\begin{equation}\label{eq:eta_last}
    \eta_2\approx \frac{\dot{d}}{V}
\end{equation}
From the definition of $d$,
\begin{equation}\label{eq:d_dot}
    \dot{d} = -\dot{\lambda}\sin\theta_g + \dot{\phi}\cos\theta_g = V(-\sin\theta\sin\theta_g + \cos\theta\cos\theta_g)
\end{equation}
By combining equations \eqref{eq:eta_first}-\eqref{eq:eta_last}, the centripetal acceleration can then be approximated as
\begin{equation}
    a_s\approx2\frac{V}{L_1}\left(\dot{d} + \frac{V}{L_1}d\right)
\end{equation}
From the laws of circular motion the angular velocity is given by
\begin{equation}\label{eq:ss_last}
    \dot{\theta}=\frac{a_s}{V}
\end{equation}
The UAV state-space model can now be defined by combining equations \eqref{eq:ss_first}-\eqref{eq:ss_last}.

\begin{definition}[UAV State-Space model]
The UAV state-space model is defined as 
\begin{equation}
    \dot{\vec{x}} = f(\vec{x}, \vec{u})
\end{equation}
where
\begin{equation}
    f(\mathbf{x}, \mathbf{u}) = \begin{bmatrix}
    V\sin\theta \\
    V\cos\theta \\
    \frac{2}{L_1}\left(\dot{d} + \frac{V}{L_1}d\right)
    \end{bmatrix}
\end{equation}
with $d$ and $\theta_g$ as defined in \eqref{eq:ss_variabels}, $\dot{d}$ as defined in \eqref{eq:d_dot} and $L_1$ as defined in \eqref{eq:ardu_l1}.
\end{definition}



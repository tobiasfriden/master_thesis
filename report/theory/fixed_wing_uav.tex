
\chapter{Fixed-wing Unmanned Aerial Vehicles}\label{cha:fixed_wing_uav}
Though not recieving the same amount of research and commercial interest as their multirotor counterparts,
fixed-wing UAVs offer a number of advantages in many usecases. In the following sections a thorough description of
general fixed-wing dynamics as well as a description of the specific platform used in this work will be presented. 

\section{Fixed-wing dynamics}
This first section will give the reader a quick overview of some common definitions and terminology. 
A general purpose 6-DOF model of a fixed-wing UAV will also be derived. 
\subsection{General definitions and terminology}
We begin with establishing some common definitions and
 terminology which will be used throughout this thesis. These definitions are 
 used in many other works related to fixed-wing aircraft, such as \cite{uav_dynamics_wind}, \cite{uav_dynamics_modeling}, 
\cite{fighter_aircraft_mpc}. 

\subsubsection{Coordinate reference frames}
Three different coordinate reference frames are used, an inertial frame, a body frame and a wind reference frame.
This is convenient since some sensors, like GPS report their values in the inertial frame
 while others, like IMU sensors report values in the body frame. Other advantages of the different references
will be made apparent when formulating the dynamic equations.

\begin{definition}[Inertial frame]
    The inertial frame, denoted with subscript $I$ is fixed relative to the earth.
    A position vector in the intertial frame is defined in the NED order as
    \begin{equation}
        \vec{p}_I = (p_N, p_E, -p_H)
    \end{equation}
    where $p_N$ points in the north direction, $p_E$ points east and $p_H$ points down towards the earth,
    in order to form a right hand positive coordinate system.
\end{definition}

\begin{definition}[Body frame]
    The body frame, denoted with subscript $B$ is fixed in the UAV center of gravity.
    A position vector in the body frame is defined as
    \begin{equation}
        \vec{p}_B = (x, y, z)
    \end{equation}
    where $x$ points forward through the UAV, $y$ points to the right and $z$ points down as shown in Figure TODO
\end{definition}

\begin{definition}[Wind reference frame]
    The wind reference frame, denoted with subscript $W$ is related to the current direction of motion
    through the air.
    A position vector in the wind reference frame is defined as
    \begin{equation}
        \vec{p}_W = (x_w, y_w, z_w)
    \end{equation}
    where $x_w$ points in the same direction as the current velocity vector $\vec{v}_I$, 
    $y_w$ points to the right of $x_w$ and $z$ points down relative $x_w$ and $y_w$.
\end{definition}

\subsubsection{Attitude representation}
The attitude of the UAV is represented by the \textit{Euler angles}. 

\begin{definition}[Euler angles]
The Euler angle vector is defined as
\begin{equation}
    \vec{\Phi}=(\phi, \theta, \psi)
\end{equation}
where the \textit{roll angle} $\phi$ is rotation around the north inertial axis, 
the \textit{pitch angle} $\theta$ is rotation around the east inertial axis and
the \textit{yaw angle} $\psi$ is rotation around the downwards inertial axis.
\end{definition}

The relationship between coordinates in the body frame and inertial frame is given
 by the rotation matrix

\begin{equation}\label{eq:r_i_b}
\mathcal{R}^I_B = \mathcal{R}^x_\phi\mathcal{R}^y_\theta\mathcal{R}^z_\psi\\
=
\begin{pmatrix}
    1 & 0 & 0 \\
    0 & \cos\phi & \sin\phi \\
    0 & -\sin\phi & \cos\phi
\end{pmatrix}
\begin{pmatrix}
    \cos\theta & 0 & -\sin\theta \\
    0 & 1 & 0 \\
    \sin\theta & 0 & \cos\theta
\end{pmatrix}      
\begin{pmatrix}
    \cos\psi & \sin\psi & 0 \\
    -\sin\psi & \cos\psi & 0 \\
    0 & 0 & 1
\end{pmatrix}
\end{equation}  
This attitude representation is not defined for $\theta=\pm\pi/2$. However, such attitudes
 were deemed very unlikely in this work as the main focus is on level flight scenarios.

\subsubsection{Aerodynamic angles}
The \textit{aerodynamic angles} relate the wind reference frame to the other coordinate frames.

\begin{definition}[Aerodynamic angles]
We first define the body frame velocity vector as 
\begin{equation}
    \vec{v}_B=(u, v, w)
\end{equation}
The \textit{angle of attack} $\alpha$ and \textit{side slip} $\beta$ are then defined as
\begin{equation}
    \alpha=\arctan\frac{w}{u}
\end{equation}
\begin{equation}
    \beta=\arcsin\frac{v}{V_a}
\end{equation}
where $V_a=\sqrt{u^2+v^2+w^2}$
\end{definition}

\subsubsection{Fixed-wing UAV}
A fixed-wing UAV is equipped with two horizontal wings that are fixed in the body frame.
In order to stay in the air, it needs to keep a minimum forward velocity
\begin{equation}
    V > V_{s}
\end{equation}
where $V_s$ is the airframe-dependent \textit{stall speed}. In order to navigate through the
air, it is equipped with some or all of the following control surfaces:
\begin{itemize}
    \item \textit{Ailerons} to control $\phi$
    \item \textit{Elevators} to control $\theta$
    \item \textit{Rudders} to control $\psi$
\end{itemize}
The UAV is also equipped with one or several propellers that are used to create the thrust which
increases the total energy of the system. These might be facing towards or against the direction of motion.

\subsection{Dynamic equations}
In the following section the dynamic equations of a general fixed-wing aircraft will be derived. The 
equations are based on the following common state representation:
\begin{equation}
    \vec{x}=
    \begin{pmatrix}
        \vec{p}_I & \vec{v}_B & \vec{\Phi} & \vec{\omega}
    \end{pmatrix}
\end{equation}
where $\vec{\omega}=(p, q, r)$ are the angular rates in the body reference frame. These can be divided
into translational and rotational dynamics which are presented separately.
\subsubsection{Translational dynamics}
The dynamic equation for the position $\vec{p}_I$ is directly given as
\begin{equation}\label{eq:dynamic_first}
    \dot{\vec{p}}_I=\mathcal{R}^I_B\vec{v}_B
\end{equation}
Furthermore, Newtons second law of motion gives 
\begin{equation}\label{eq:uav_force}
    \vec{F}_{tot}=m\dot{\vec{v}}_B + \vec{\omega}\times m\vec{v}_B 
\end{equation}
where $m$ is the UAV mass and $\vec{F}_{tot}$ is the sum of all forces working on the UAV.
These can be divided into
\begin{equation}
    \vec{F}_{tot}=\vec{F}_g+\vec{F}_{aero}+\vec{F}_{thr}
\end{equation}
where $\vec{F}_g$ is the gravitational force, $\vec{F}_{aero}$ is the aerodynamic force and
$\vec{F}_{thr}$ is the thrust force. The thrust force is assumed to only work in the same direction
as the x-axis in the body frame, so that
\begin{equation}
    \vec{F}_{thr, B}=(T, 0, 0)
\end{equation}
\subsubsection{Rotational dynamics}
The dynamic equations for the Euler vector $\vec{\Phi}$ are given by
\begin{equation}
    \begin{pmatrix}
        \dot{\phi} \\
        \dot{\theta} \\
        \dot{\psi}
    \end{pmatrix}
    =
    \begin{pmatrix}
        1 & -\sin\phi\tan\theta & \cos\phi\tan\theta \\
        0 & \cos\phi & \sin\phi \\
        0 & -\sin\phi\sec\theta & \cos\phi\sec\theta
    \end{pmatrix}
    \begin{pmatrix}
        p \\
        q \\
        r
    \end{pmatrix}
\end{equation}
By again using Newtons second law for the moment we derive
\begin{equation}
    \vec{M}=\vec{I}\dot{\vec{\omega}} + \vec{\omega}\times\vec{I}\vec{\omega}
\end{equation}
The moment $\vec{M}$ is assumed to mainly come from aerodynamics.
\subsubsection{Aerodynamics}
The aerodynamic forces and moments generated require some further analysis. The aerodynamic
force $\vec{F}_{aero}$ can be written in the wind reference frame as
\begin{equation}
    \vec{F}_{aero}=
    \begin{pmatrix}
        -D \\
        Y \\
        -L
    \end{pmatrix}=
    \begin{pmatrix}
        \bar{q}SC_D \\
        \bar{q}SC_Y \\
        \bar{q}SC_L
    \end{pmatrix}
\end{equation}
where $\bar{q}=\frac{1}{2}\rho(h)V^2$ is the free-stream dynamic pressure, $S$ is the 
wing surface area and $C_D$, $C_Y$ and $C_L$ are dimensionless constants.
$D$, $Y$ and $L$ are called the \textit{drag}, \textit{sideforce} and \textit{lift} respectively. 
The corresponding coefficients are dependent on a number of variables, such as the aerodynamic angles
and airframe dependent parameters \cite{fighter_aircraft_mpc}.\\

The aerodynamic moments are usually defined in the body reference frame as
\begin{equation}\label{eq:dynamic_last}
    \vec{M}_{aero}=
    \begin{pmatrix}
        M_x \\
        M_y \\
        M_z
    \end{pmatrix}=
    \begin{pmatrix}
        \bar{q}SbC_l \\
        \bar{q}SbC_m \\
        \bar{q}SbC_n
    \end{pmatrix}
\end{equation}
where $b$ is the wing-span of the aircraft and $C_l$, $C_m$ and $C_n$ are dimensionless 
constants mainly dependent on the aerodynamic angles \cite{uav_dynamics_wind}. \\

\subsubsection{Wind effects}
The dynamic equations derived in \eqref{eq:dynamic_first}-\eqref{eq:dynamic_last} are under
the assumption that the air through which the UAV is travelling is at rest relative to the earth.
During real flights, wind will cause the air to move which has non-negligible effects on the UAV.
The wind vector is often defined in the inertial frame as
\begin{equation}
    \vec{w}=(w_N, w_E, w_H)
\end{equation}
$\vec{w}$ can be decomposed to
\begin{equation}
    \vec{w}=\vec{w}_{m}+\vec{w}_s
\end{equation}
where $\vec{w}_m$ is the mean wind and $\vec{w}_s$ is some stochastic process \cite{spline_trajectory}. In this work,
we only consider the mean wind.
The mean wind magnitude $W=|\vec{w}|$ varies with altitude as
\begin{equation}
    W(h)=W_{10}(\frac{h}{h_{10}})^a
\end{equation}
where $W_{10}$ is the wind measured at $h=10$ meters and $a$ is the Hellman exponent, which depends on
the shape and coastal location of the underlying terrain \cite{uav_dynamics_wind}. The wind can be 
incorporated in the dynamic equations of motion by extending \eqref{eq:dynamic_first} to
\begin{equation}
    \dot{\vec{p}}=\mathcal{R}^I_B\vec{v}_B+\vec{w}
\end{equation}
Also, we need to extend \eqref{eq:uav_force} by replacing $\vec{v}_B$ with
\begin{equation}
    \vec{v}_g = \vec{v}_B + \vec{w}_B
\end{equation} 
where $\vec{v}_g$ is the body-fixed velocity relative the ground and $\vec{w}_B$ is the
wind vector in the body frame.

\subsubsection{Complete dynamics model}
The complete dynamic model is given by combining the above equations. It will not be 
written out here, but is given in e. g. \cite{uav_dynamics_wind}. The dynamics are clearly nonlinear and quite complex. Extensive modeling and experimentation, such as described in \cite{aerodynamic_parameters}
is needed to identify the different parameters which describe the behaviour of a specific airframe.

It is often interesting to study the \textit{longitudal} and \textit{lateral} dynamics
separately, where \textit{longitudal} denotes movement in the vertical plane and
\textit{lateral} denotes movement out of the vertical plane. The decoupled state
vectors are given by
\begin{equation}
    \vec{x}_{lng}=(p_N, p_H, u, w, \theta, q)
\end{equation}
and
\begin{equation}
    \vec{x}_{lat}=(p_E, v, \phi, \psi, p, r)
\end{equation}
By introducing the polar inertial components $(V, \gamma)$ and writing the dynamic equations
in the wind reference frame the longitudal dynamics become
\begin{align}
    \dot{V}&=\frac{1}{m}[T\cos\alpha-D-mg\sin\gamma] \\
    \dot{\gamma}&=\frac{1}{mV}[T\sin\alpha + L - mg\cos\gamma] \\
    \dot{q}&=\frac{M_y}{I_{yy}} \\
    \dot{\alpha}&=\dot{\theta}-\dot{\gamma}=q-\frac{1}{mV}[T\sin\alpha + L - mg\cos\gamma]
\end{align}
Assuming level flight in the longitudal direction, i. e. $u$, $V$ and $p_H$ are constant, the lateral dynamics can be formulated as
\begin{align}
    \dot{p}_E &= u\sin\psi + v\cos\phi\cos\psi \\
    \dot{\psi} &= r\cos\phi \\
    \dot{\phi} &= p \\
    \dot{v} &= \frac{Y_B}{m} + g\sin\phi - ru \\
    \dot{p} &= \frac{I_{zz}M_x + I_{xz}M_z}{I_{xx}I_{zz} - I^2_{xz}} \\
    \dot{r} &= \frac{I_{xz}M_x + I_{xz}M_z}{I_{xx}I_{zz} - I^2_{xz}} \\
\end{align}
\section{ArduPlane autopilot}
The ArduPlane autopilot is an open source autopilot for fixed-wing UAVs \cite{arduplane}. 
It contains high-level controllers for navigation, velocity and altitude control as well as 
low level logic to command the attitude and throttle of the vehicle. In the following section
the underlying theory of the relevant control loops for this thesis will be presented.

\subsection{Navigation control loop}
The ArduPlane autopilot the $L_1$ control law described in \cite{arduplane_l1} for navigation.
The goal of the control loop is to follow a straight line from a start coordinate $(x_s, y_s)$ to a goal
coordinate $(x_g, y_g)$. This is obtained by aiming towards a point $P$ which is located at a
fixed distance $L_1$ from the UAV. 
The logic behind the controller is illustrated in Figure~\ref{fig:ss_defs},
where $(x, y)$ is the UAV position and $\psi$ is the UAV yaw angle.
\begin{figure}[htb]
    \begin{center}
    \tikzstyle{point} = [
        circle,
        minimum width=3.5pt,
        inner sep=0,
        fill=black
    ]
    \tikzstyle{my_v} = [
        ->,
        line width = 1.2pt
    ]
    \begin{tikzpicture}[scale=0.85]
        %\draw[help lines](0,0) grid (10,10);
        \coordinate (origin) at (1,1);
        \coordinate (drone) at (2,6);
        \coordinate (goal) at (9,9);
        \coordinate (ref) at (7,7);
        
        \node[anchor=east] at (drone) {$(x, y)$};
        \draw[my_v] (drone) -- node[above, near end, anchor=south east]{$\vec{v}$} ++ ({atan(2)}:2.5);

        \draw[->] ([yshift=0.7cm]drone) arc(90:{atan(2)}:0.7) node[above,midway]{$\psi$};
        \draw[dashed] (drone) --++ (0,1);
        \draw[->] (drone) ++ (0.5,1) arc({atan(2)}:{atan(0.2)}:{sqrt(1.25)}) node[midway,anchor=223]{$\eta$};
        
        \node[anchor=north west] at (origin) {$(x_s, y_s)$};

        \node[anchor=north west] at (goal) {$(x_g, y_g)$};
        
        \draw (drone) -- node[above,sloped]{$L_1$} (ref);
        \node[point] at (ref) {};
        \node[anchor=north west] at (ref) {$P$};
        
        \draw[my_v] (origin) -- node[above, at end]{$\vec{g}$} (goal);
        \draw[my_v] (origin) -- node[left]{$\vec{r}$} (drone);

    \end{tikzpicture}
        
    \caption{$L_1$ controller logic}
    \label{fig:ss_defs}
    \end{center}
\end{figure}

In the ArduPilot implementation, the distance $L_1$ is calculated as
\begin{equation}\label{eq:ardu_l1}
    L_1=\begin{cases}
        \frac{1}{\pi}\zeta\Delta TV & \mbox{if}\quad |\frac{1}{\pi}\zeta\Delta TV|>|\vec{r}-\vec{g}| \\
        |\vec{r}-\vec{g}| & \mbox{otherwise}
    \end{cases}
\end{equation}
where $V=|\vec{v}|$, $\zeta$ is the damping factor and $\Delta T$ is the update period of the controller \cite{arduplane_l1}.

 In each time step, the control law corresponds to following a circular segment with radius
 \begin{equation}
    R=\frac{L_1}{2\sin\eta}
 \end{equation}
 which is tangent to $\vec{v}$ in $(x,y)$.
 $\eta$ is defined as the angle between the UAV velocity vector $\vec{v}$ and the line from the UAV to $P$.
This circular segment is followed by issuing a lateral acceleration command
\begin{equation}
    a_{cmd}=2\frac{V^2}{L_1}\sin\eta
\end{equation}
The lateral acceleration command is translated to a desired roll angle
\begin{equation}
    \phi_{cmd}=\cos\theta\tan^{-1}(a_{cmd}/g)
\end{equation}
where $g$ is the gravitational constant. The low-level attitude controller is then used to track the desired roll.

\subsection{Altitude and velocity control loop}
ArduPlane uses a combined control loop to handle both desired altitude and velocity, called 
TECS (Total Energy Control System). This controller is based on the total energy of the UAV,
which is defined as
\begin{equation}
    E_T=\frac{1}{2}mV^2 + mgh
\end{equation}
where $h$ is the altitude relative to the takeoff point. The total energy rate is derived
by taking the derivative with respect to time as
\begin{equation}
    \dot{E}_T=mV\dot{V} + mg\dot{h}
\end{equation}
The specific energy rate is then
\begin{equation}
    \dot{E}_S = \frac{\dot{E}_T}{mgV} = \frac{\dot{V}}{g} + \frac{\dot{h}}{V} = \frac{\dot{V}}{g} + \sin\gamma
\end{equation}
If $\gamma$ is small, we get
\begin{equation}
    \dot{E}_S\approx\frac{\dot{V}}{g} + \gamma
\end{equation} 
The longitudinal aircraft dynamics give
\begin{equation}
    T-D=\frac{\dot{V}}{g} + \gamma
\end{equation}
Thus, by increasing the thrust
energy is added to the system. By changing the pitch angle using the elevators, the balance 
between kinetic and potential energy can be modified. 

\subsection{Mission representation and flight modes}
A \textit{mission} $\mathcal{M}$ is defined as 
\begin{equation}
    \mathcal{M} = \{\vec{p}_1, \hdots, \vec{p}_n\}
\end{equation}
i. e. a sequence of $n$ \textit{waypoints} represented in the inertial frame as 
\begin{equation}
    \vec{p}=(p_N, p_E, -p_H, c_{wp})
\end{equation}
where $c_{wp}$ represents the waypoint command. There are many different waypoint commands available
in ArduPlane, but this work will be focused on 
\begin{equation}
    c_{wp}\in \{Waypoint, Takeoff, Land\}
\end{equation}
\subsubsection{Waypoint mode}
In the \textit{waypoint} mode uses the navigation control loop
to navigate along the line from $\vec{p}_i$ to $\vec{p}_{i+1}$. When $\vec{p}_{i+1}$ is reached, the flight mode
is updated depending on the next $c_{wp}$. The next waypoint is assumed to be reached when
\begin{equation}
    |\vec{p}_{UAV}-\vec{p}_{wp}| < R_{wp}
\end{equation}
where $R_{wp}$ is defined by the user.
\subsubsection{Takeoff mode}
In \textit{takeoff} mode, the plane will attempt to reach the given altitude $h_{takeoff}$, while
keeping a constant heading given by the launch direction. The takeoff can be triggered either
by throwing the UAV, launching it from a catapult or on a runway \cite{ardupilot_takeoff}.

\subsubsection{Land mode}
In \textit{Land} mode, the plane will attempt to land at a given coordinate. The landing approach 
is divided into two different stages, the \textit{approach} stage and \textit{flare} stage.

During the approach stage, the UAV tries to accomplish the commanded \textit{glide slope}, which is
dependent on the previous waypoint position relative to the landing point. When the altitude decreases
below $h_{flare}$, it enters the flare stage which means the throttle is completely turned off. 
During this stage the UAV will simply try to hold a target descent rate $\dot{h}_{flare}$ which is
defined by the user \cite{ardupilot_land}.
\chapter{Wind field definition and estimation}\label{cha:wind}
\section{Wind field definition}
The wind field is commonly defined as a time and spatially dependent vector field
\begin{equation}
    \vec{w}(p_N,p_E,p_H,t)=
    \begin{bmatrix}
        w_N(p_N,p_E,p_H,t) \\
        w_E(p_N,p_E,p_H,t) \\
        w_H(p_N,p_E,p_H,t)
    \end{bmatrix}
\end{equation}
In this thesis, the vertical component $w_H$ will be neglected and the wind vector is written as 
\begin{equation}
    \vec{w}=W\begin{bmatrix}
        \cos\psi_w\\
        \sin\psi_w
    \end{bmatrix}
\end{equation}
where $W$ is the wind magnitude and $\psi_w$ is the wind direction. The dependency on position and time will from now on not be written out explicitly for simplicity.
The wind field can be decomposed as
\begin{equation}
    \vec{w} = \bar{\vec{w}} + \vec{w}_s
\end{equation}
where $\bar{\vec{w}}$ is the mean wind field and $\vec{w}_s$ is described by some stochastic process.

\subsection{Wind gradient}
The wind magnitude $W$ is dependent on the altitude above ground $h$. A simplified model of this relationship is 
\begin{equation}
    W(h)=W_{0}\left(\frac{h}{h_0}\right)^a
\end{equation}
where $W_{0}$ is wind measured at a reference height $h_0$, often 10 meters and $a$ is the Hellman exponent. $a$ is dependent on parameters such as the 
roughness of the terrain and if the location is coastal or not \cite{wind_hellman}.
\subsection{Turbulence}
The stochastic components of wind is often modeled using \textit{Dryden's Gust Model} which is stochastic process with spectral density 
\begin{equation}
    \Phi_u(\Omega)=\sigma_u^2\frac{L_u}{\pi}\frac{1}{1+(L_u\Omega)^2}
\end{equation}
\begin{equation}
    \Phi_w(\Omega)=\sigma_w^2\frac{L_w}{\pi}\frac{1 + 3(L_w\Omega)^2}{(1+(L_w\Omega)^2)^2}
\end{equation}
where $u$ is the horizontal component and $w$ is the vertical. For altitudes below 1000 feet, the length scale of the vertical gust is 
$L_w=h$ and the intensity is $\sigma_w=0.1W_{20}$ where $W_{20}$ is the wind magnitude at 20 feet. The horizontal gust length and intensity are related to the vertical as 
\begin{equation}
    \frac{L_u}{L_w}=\frac{1}{(0.177 + 0.000823h)^{1.2}}
\end{equation}
\begin{equation}
    \frac{\sigma_u}{\sigma_w}=\frac{1}{(0.177 + 0.000823h)^{0.4}}
\end{equation}
where $h$ is measured in feet \cite{wind_direct_computation}.

\section{Wind estimation}
Wind field estimation techniques are important for efficiently handling the effects of winds on UAVs. 
Fixed-wing UAVs are often equipped with a \textit{pitot-tube} sensor which makes it possible to measure the 
true airspeed (TAS) of the UAV through the surrounding air.
\subsection{Direct computation of wind field}
If the inertial velocity vector $\vec{v}_I$ can be measured, e. g. with the GPS system of the UAV the wind vector can be 
computed directly as 
\begin{equation}
    \vec{w} = \vec{v}_I - (\mathcal{R}^I_B)^{-1}\vec{v}_B
\end{equation}
where $\mathcal{R}^I_B$ is given by Equation \eqref{eq:r_i_b}. Assuming level flight, it is shown in \cite{wind_direct_computation} that 
the measurement error is
\begin{equation}
    e^2=\sigma_{\dot{p}_N}^2+\sigma_{\dot{p}_E}^2+\sigma_{\dot{p}_H}^2+\sigma_{V_a}^2 + V_a^2(\sigma_{\theta}^2+\sigma_{\alpha}^2+\sigma_{\beta}^2+\sigma_{\psi}^2)
\end{equation}
In standard unaided GPS systems, the measurement error is approximately 0.1 m/s. Assuming the measurement error of $V_a$ is 0.2 m/s and angles 
can be measured up to $1\degree$ precision, the error becomes $e^2=0.07+0.0012V_a^2$. If the airspeed is 16 m/s this corresponds to a measurement error of 
$e=0.61$ m/s.
\subsection{Estimation using Extended Kalman Filter}\label{sec:wind_ekf}
A more robust approach is to use an Extended Kalman Filter (EKF) to measure vehicle states. These are 
commonly used in autonomous systems to fuse measurements from many different sensors such as GPS, Inertial Measurement Units (IMU) and barometers. 
A thorough reference on the underlying theory of EKFs is given in \cite{sensor_fusion}.

The ArduPlane EKF implementation uses 24 different states such as attitude, velocity, position, sensor biases and wind. The different process models and 
measurement equations are presented in \cite{px4_ecl_ekf}.
\chapter{Wind field definition and estimation}\label{cha:wind}

\subsection{Wind altitude dependency}
The wind magnitude $\windspd$ is dependent on the altitude above ground $h$. A simplified model of this relationship is 
\begin{equation}
    \windspd(h)=W_{0}\left(\frac{h}{h_0}\right)^a
\end{equation}
where $W_{0}$ is wind measured at a reference height $h_0$ - often 10 meters - and $a$ is the Hellman exponent. $a$ is dependent on parameters such as the 
roughness of the terrain and if the location is coastal or not \cite{wind_hellman}.
\subsection{Turbulence}
The stochastic components of wind is often modeled using \textit{Dryden's Gust Model} which is stochastic process with spectral density 
\begin{equation}
    \Phi_u(\Omega)=\sigma_u^2\frac{L_u}{\pi}\frac{1}{1+(L_u\Omega)^2}
\end{equation}
\begin{equation}
    \Phi_w(\Omega)=\sigma_w^2\frac{L_w}{\pi}\frac{1 + 3(L_w\Omega)^2}{(1+(L_w\Omega)^2)^2}
\end{equation}
where $u$ is the horizontal component and $w$ is the vertical. For altitudes below 1000 feet, the length scale of the vertical gust is 
$L_w=h$ and the intensity is $\sigma_w=0.1W_{20}$ where $W_{20}$ is the wind magnitude at 20 feet. The horizontal gust length and intensity are related to the vertical as 
\begin{equation}
    \frac{L_u}{L_w}=\frac{1}{(0.177 + 0.000823h)^{1.2}}
\end{equation}
\begin{equation}
    \frac{\sigma_u}{\sigma_w}=\frac{1}{(0.177 + 0.000823h)^{0.4}}
\end{equation}
where $h$ is measured in feet \cite{wind_direct_computation}.

\section{Wind estimation}
Wind field estimation techniques are important for efficiently handling the effects of winds on \acp{uav}. 
This estimation is often performed by comparing the inertial velocity of the \ac{uav} to the velocity relative to the air.
\subsection{Airspeed measurement}
Fixed-wing \acp{uav} are often equipped with a \textit{pitot-tube} sensor to measure airspeed.
Such a sensor consists of a metallic tube together with a sensor which measures the dynamic pressure $\Delta P$ of 
the air which flows through the tube. This measurement is related to the velocity of the air flow $V_{\text{pitot}}$ through 
Bernoulli's Equation as 
\begin{equation}
    V^2_{\text{pitot}} = K\frac{2\Delta P}{\rho}
\end{equation}
where $\rho$ is the density of the air. $K$ is a correction factor which compensates for calibration errors and 
generalizations such as assuming a perfect gas and constant temperature. Assuming that the sensor is mounted 
along the $x$ axis in the body frame, the airspeed is related to $V_{\text{pitot}}$ in \cite{pitot} as
\begin{equation}
    \airspd^2=\frac{V^2_{\text{pitot}}}{\cos\alpha\cos\beta}=\frac{2K\Delta P}{\rho\cos\alpha\cos\beta}
\end{equation}  

\subsection{Direct computation of wind field}
If the inertial velocity vector $\vel$ can be measured, \eg\ with the \ac{gps} of the \ac{uav} the wind vector can be 
computed directly as 
\begin{equation}
    \windvec = \vel - \airvec
\end{equation}
which is given by the relationship illustrated in Figure \ref{fig:wind_triangle}. Assuming level flight, it is shown in \cite{wind_direct_computation} that 
the measurement error is
\begin{equation}
    e^2=\sigma_{\dot{x}_N}^2+\sigma_{\dot{y}_E}^2+\sigma_{\dot{z}_H}^2+\sigma_{\airspd}^2 + \airspd^2(\sigma_{\theta}^2+\sigma_{\alpha}^2+\sigma_{\beta}^2+\sigma_{\psi}^2)
\end{equation}
In standard unaided \ac{gps} systems, the measurement error is approximately 0.1 m/s. Assuming the measurement error of $\airspd$ is 0.2 m/s and angles 
can be measured up to $1\degree$ precision, the error becomes $e^2=0.07+0.0012V_a^2$. If the airspeed is 16 m/s this corresponds to a measurement error of 
$e=0.61$ m/s.
\subsection{Estimation using Extended Kalman Filter}\label{sec:wind_ekf}
A more robust approach is to use an \ac{ekf} to measure vehicle states. These are 
commonly used in autonomous systems to fuse measurements from many different sensors such as a \ac{gps}, \ac{imu} and barometer. 
A thorough reference on the underlying theory of \acp{ekf} is given in \cite{sensor_fusion}.

The ArduPlane \ac{ekf} implementation uses 24 different states such as attitude, velocity, position, sensor biases and wind. The different process models and 
measurement equations are presented in \cite{px4_ecl_ekf}.
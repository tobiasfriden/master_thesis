\chapter{Motion planning}\label{cha:motion_planning}
In this chapter the necessary background regarding motion planning is 
introduced. In this thesis, motion planning is defined as the task of finding a
path from a starting state to a goal state which fulfills a given set of constraints.
These constraints might include differential constraints of the system and obstacle avoidance among others.

\section{Motion planning with differential constraints}
\subsection{General definitions and terminology}
We begin with introducing general definitions and terminology that are used to
describe motion planning in this thesis.
\subsubsection{Motion planning terminology}
First we shall define some common terms used in motion planning.

\begin{definition}[State and action spaces]
    We define the \textit{state space} $\mathcal{X}$ and \textit{action space}
    $\mathcal{U}$ as the set of obtainable states $x$ and available actions $u$ for the
    studied system.
    $\mathcal{X}$ can be further divided into
    \begin{equation}
        \mathcal{X} = \mathcal{X}_{free} + \mathcal{X}_{obs}
    \end{equation}
    where $\mathcal{X}_{obs}$ are states which contain some kind of obstacle. 
\end{definition}

\begin{definition}[Motion plan]
    A motion plan is defined as a sequence of states
    \begin{equation}
        \{x(t_1),\hdots, x(t_n)\}\in\mathcal{X}_{free}
    \end{equation}
    and actions
    \begin{equation}
        \{u(t_1),\hdots,u(t_n)\}\in\mathcal{U}
    \end{equation}
    which takes the system from a specified initial state $x(t_1)=x_S$ to a 
    goal state $x(t_n)=x_G$ while fulfilling
    \begin{equation}
        x(t_{i+1})=x(t_i) + \int_{t_i}^{t_{i+1}} f(x, u) dt
    \end{equation}
    where $f(x, u)$ is the \textit{transition function}.
\end{definition}

\subsubsection{Differential constraints}
Differential constraints restrict the set of possible actions and states that 
the system can obtain. An important class of systems under differential constraints 
are \textit{non-holonomic} systems.

\begin{definition}[Non-holonomic system]
    In a \textit{non-holonomic} system, the current state $x(t)$ is dependent 
    on in which order the actions $u(t_i),\quad t_i<t$ where performed.
\end{definition}

A formal definition and extensive discussion of non-holonomic systems is given 
in \cite[Chapter~15]{planning_algorithms}. Systems only capable of motion in a direction 
dependent on the current state, such as cars and fixed-wing UAVs belong to this class of systems.

\subsection{Sampling based motion planning}
Both $\mathcal{X}$ and $\mathcal{U}$ are generally continous, and need to be sampled 
in some way. This means that the resulting path will only be \textit{resolution complete}, i. e. 
the optimality of the plan will depend on the sampling resolution $d$. In sampling based
motion planning a \textit{reachability graph} is commonly used.

\begin{definition}[Reachability graph]
    Given a starting state $x_0(t_0)\in\mathcal{X}_d$, we define the \textit{reachable set} 
    $R(x_0, \mathcal{U}_d)$ as the set of states which are reached by applying any action $u\in\mathcal{U}_d$.
    By incrementally calculating the reachable set for each $x\in R(x_0, \mathcal{U}_d)$ we create the \textit{reachability tree} $\mathcal{T}_r(x_0, \mathcal{U}_d)$.
    The reachability tree is a directed graph where each vertex consists of a state $x$ which is reachable from $x_0$ by applying some action sequence 
    $\{u_1,\hdots,u_n\}\in\mathcal{U}_d$. By pruning any duplicate states from $\mathcal{T}_r$ we finally reach the \textit{reachability graph} $\mathcal{G}_r(x_0, \mathcal{U}_d)$
\end{definition}

\subsubsection{Forward simulation}
The next state in $G_r$ given a specified input action $u$ is obtained by 
integrating the transition function $f(x, u)$ on $[0, \Delta t]$. In practice this 
integral is calculated using some numerical approximation method. A common choice is the 
fourth-order \textit{Runge-Kutta integration method}
\begin{equation}
    x(\Delta t)\approx x(0) + \frac{\Delta t}{6}(w_1 + 2w_2 + 2w_3 + w_4)
\end{equation}
where
\begin{align}
\begin{split}
    w_1 &= f(x(0), u) \\
    w_2 &= f(x(0) + \frac{1}{2}\Delta t w_1, u) \\
    w_3 &= f(x(0) + \frac{1}{2}\Delta t w_2, u) \\
    w_3 &= f(x(0) + \Delta t w_3, u)
\end{split}
\end{align}
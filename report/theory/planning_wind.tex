\chapter{Motion planning using waypoint optimization}\label{cha:motion_planning_fw}
\section{Waypoint sampling}
The desired output of the motion planner in this thesis is a waypoint sequence $\mathcal{M}$, as defined in section \ref{sec:mission},
which takes the \ac{uav} from a desired initial state $(x_{N,0},y_{E,0},\psi_0)$ to a goal state $(x_{N,g},y_{E,g},\psi_g)$. Moreover, physical constraints of the \ac{uav} and wind should be taken into account. This formulation is well aligned with \textit{input sampling} methods such as Hybrid $A^*$ \cite{hybrid_astar}. 
In such methods, the reachability graph is created by forward simulation of the transition function $f(x, u)$ using input values $u$ sampled from a 
set $\inputs$. 

\subsection{State and input set definition}
Based on the kinematic model in Equation \eqref{eq:traj_model} the state vector is defined as
\begin{equation}
    x=(x_N, y_E, \psi)
\end{equation}
The input is defined as 
\begin{equation}
    u=\vec{p}_{i+1}-\vec{p}_i\equiv(\Delta x_N, \Delta y_E)
\end{equation}
\ie\ the coordinates of the next waypoint relative to the current, specified in the inertial frame.
\subsection{State transition function}
The definition of $u$ combined with the trajectory-following controller, derived in Section \ref{sec:traj_controller}, 
leads to a model of the closed-loop system. Using this a model during forward simulation implies, assuming it is valid, that controller will be able to follow the planned waypoint sequence. 

The desired \ac{cog} to follow a straight line from $(0,0)$ to $(\Delta y_E, \Delta x_N)$ is
\begin{equation}
    \psi_u=\atan2(\Delta y_E,\Delta x_N).
\end{equation}
If roll dynamics are neglected, the commanded turn-rate is obtained by inserting the roll command from Equation \eqref{eq:roll_cmd} into Equation \eqref{eq:traj_model}, which gives 
\begin{equation}
    \dot{\psi}_{\text{cmd}}=\frac{a_{\text{cmd}}}{\airspd}
\end{equation}
where $a_{\text{cmd}}$ is given by Equation \eqref{eq:lat_acc} with $\eta$ as defined in \eqref{eq:eta_first}--\eqref{eq:eta_last}.
However, for a real \ac{uav} the magnitude of the turn-rate is limited by some $\dot{\psi}_{\text{max}}$. The actual value of $\dot{\psi}$ is hence
\begin{equation}\label{eq:saturation}
    \dot{\psi}=\begin{cases}
        \dot{\psi}_{\text{cmd}} & |\dot{\psi}_{\text{cmd}}| \leq \dot{\psi}_{\text{max}} \\
        \text{sgn}(\dot{\psi}_{\text{cmd}})\dot{\psi}_{\text{max}} & \text{otherwise}
    \end{cases}
\end{equation}
Finally, the kinematic model of the closed-loop system becomes
\begin{equation}\label{eq:closed_loop}
    \dot{x}=f(x,u)=
    \begin{bmatrix}
        V_N\\
        V_E\\
        \dot{\psi}
    \end{bmatrix}
\end{equation}

\section{Input set generation}\label{sec:motion_prims_wind}
An input set $\inputs$ is a subset of a motion primitive set $\mathcal{P}$ introduced in Section \ref{sec:motion_prim} since it contains only of pre-computed inputs but no state trajectories. Therefore, 
the heading-change method introduced in \cite{Bergman_lic} to generate $\mathcal{P}$ can also be applied when generating $\inputs$. The resulting inputs will consist
of waypoints that result in a desired change of direction while taking \ac{uav} kinematics, wind and tracking performance of the controller into account.

\subsection{Optimal control formulation}
The input set is generated by solving the optimal control problem
\begin{subequations}
    \label{eq:opt_problem_mp_uav}
    \begin{alignat}{3}
    &\min_{x(t),u,T}        &\qquad& J=\Phi(x(T),u) + \int_{0}^{T}\airspd dt & \\
    &\text{subject to} & & \psi(0)=\wca &\\
    & & & |\cog(x(T))-\Delta\cog| \leq \Delta\psi_{\text{min}} &\\
    & & & \dot{x}=f(x(t), u) &\\
    & & & x(t)\in\states& \\
    & & & u\in\actions &
    \end{alignat}
\end{subequations}
for different values of $\windvec$ and direction change $\Delta\cog$. To increase the feasible region, the constraint on $\cog(x(T))$ is relaxed to allow values in a region around the desired $\Delta\cog$ 
instead of a strict equality constraint. The closed-loop kinematic model \eqref{eq:closed_loop} depends on wind, which has to be taken into account when generating $\inputs$. This dependency as well as other relevant properties of \eqref{eq:opt_problem_mp_uav} are discussed in the sections below.

\subsubsection{Discretization of the wind direction}
During motion planning in wind, the heading relative to the wind $\psi_r=\psi-\winddir$ might take on any value.
In practice, this implies that inputs must be generated for a set of discrete wind directions $\{\psi_{r,0},\hdots,\psi_{r,n}\}$ which cover 360 degrees. 
Given a relative heading
\begin{equation}
    \psi_r: \psi_{r,i}<\psi_r<\psi_{r,i+1}
\end{equation}
the input $u\in\inputs$ selected by the planning algorithm was generated for $\psi_{r,i}$ or $\psi_{r,i+1}$, as illustrated in Figure \ref{fig:discrete_wind}. The discretization interval $|\psi_{w,i+1}-\psi_{w,i}|$ has to be sufficiently small
in order to secure good tracking performance of the closed-loop system.

\begin{figure}
    \centering
    \begin{tikzpicture}
        \Drone{0}{0}{45};
        \draw[my_v](0,0) -- node[at end, above]{$\vel$} (70:1.5);
        \draw[my_v](0,0) -- node[at end, above]{$\windvec$} (15:1.5);
        \draw[my_v] (70:0.5) arc(70:15:0.5) node[midway, anchor = south west, inner sep=1pt]{$\psi_r$};

        \draw[my_v] (2.25,0.5) arc(120:60:1);

        \draw[dotted](4,-1) -- (4,2);
        \draw[dotted](4,-1) -- node[at end, above]{$\psi_{r,1}$} ++  (75:3);
        \draw[dotted](4,-1) -- node[at end, above]{$\psi_{r,2}$} ++ (60:3);
        \draw[dotted](4,-1) -- node[at end, above]{$\psi_{r,3}$} ++ (45:3);
        \draw (4,-1) --++ (55:3);

        \draw[my_v](4,-1) ++ (90:1) arc(90:55:1) node[midway, anchor = south, inner sep=2pt]{$\psi_r$};
    \end{tikzpicture}
    \caption{When selecting inputs during online planning, the discrete relative wind direction $\psi_{r,i}$ which is closest to the actual $\psi_r$ is selected, in this case $i=2$.}
    \label{fig:discrete_wind}
\end{figure}

\subsubsection{Planning with \ac{cog} instead of heading}
As shown in Section \ref{sec:straight_path_wind}, the heading required to follow a line-segment is dependent on the current wind $\windvec$. 
Therefore constraints related to direction change in Equation~\eqref{eq:opt_problem_mp_uav} are formulated in terms of $\cog$ as defined in Equation~\eqref{eq:cog}.
A direct consequence is that the initial value of $\psi$ when generating inputs should be 
set to $\wca$ defined in Equation \eqref{eq:wca}, as this corresponds to an initial $\cog$ of $0\degree$.

\subsubsection{Cross-track error penalty}
The cross-track error at the end of a line-segment can be calculated as 
\begin{equation}
    d(x(T), u) = x_{N}(T)\sin\psi_u-y_{E}(T)\cos\psi_u
\end{equation}
If $d(x(T), u)\neq0$ the initial cross-track error for the next line-segment will be non-zero. 
Since the closed-loop system is used when expanding the graph, a small initial error can be mitigated, but large errors should be discouraged as they result in unpredictable behaviour of the controller.
The cross-track error penalty is defined as
\begin{equation}\label{eq:phi_d}
    \Phi(x(T), u)=\lambda_d\max(|d(x(T), u)|-d_{\text{min}},0)
\end{equation}
and is included in the optimization objective \eqref{eq:opt_problem_mp_uav}. 
This term is, by construction, zero when the final cross-track error is below some acceptable threshold $d_{\text{min}}$. In this case, only the trajectory length is penalized. 
The penalty for larger cross-track errors is tuned by the scaling factor $\lambda_d>0$.

\subsection{Solving the optimal control problem}\label{sec:solve_opt_ctrl}
Methods commonly used to solve optimal control problems include \textit{multiple shooting} 
and \textit{direct collocation} \cite{multiple_shooting}. 
However, the following properties of \eqref{eq:opt_problem_mp_uav} makes it hard to solve with such methods:
\begin{enumerate}
    \item The closed-loop system is highly non-linear, especially when including the saturation from Equation \eqref{eq:saturation}.
    \item In optimal control problems the input $u(t)$ can normally be chosen freely from $\actions$ for each time-step, while 
    in this formulation the input is forced to be a constant $u(t)=u$, $0<t<T$.
\end{enumerate}
The second property implies that when transformed to a Nonlinear Program using \eg\ multiple shooting,
the optimization variables corresponding to $x(t)$ in each time-step all depend on the same constant $u$. In this sense the resulting formulation is 
more closely related to a \textit{direct shooting} problem, which are known to be less linear and thus harder to solve \cite{multiple_shooting}.

\subsubsection{Derivative-free Optimization}
Since all properties of the solution of Equation \eqref{eq:opt_problem_mp_uav} are dependent on the choice of the input $u$, different solutions can be studied by simulating
the closed-loop system for different choices of $u$. Choices of $u$ which lead to solutions that violate the constraints 
can easily be pruned. A number of solutions with different characteristics, for a desired course change $\Delta\cog=90\degree$, are illustrated in Figure~\ref{fig:opt_contour_1}-\ref{fig:opt_contour_4}. 
The brighter color defines a higher value of the objective function $J$, and the green and red arrows indicate the initial and final positions of the \ac{uav}, respectively. The wind was defined as $\winddir=0\degree$ and $\windspd=5$ m/s.
As can be seen, the feasible region is non-convex but there is a clear global optimum.

\begin{figure}[H]
    \centering
    \includegraphics[width=.8\linewidth]{J_sim_90_140}
    \caption{$u=(90,140)$: Infeasible solution due to incorrect final $\cog$.}
    \label{fig:opt_contour_1}
\end{figure}

\begin{figure}[H]
    \centering
    \includegraphics[width=.8\linewidth]{J_sim_50_90}
    \caption{$u=(50,90)$: Sub-optimal solution due to large final cross-track error, $J=891$.}
    \label{fig:opt_contour_2}
\end{figure}

\begin{figure}[H]
    \centering
    \includegraphics[width=.8\linewidth]{J_sim_10_280}
    \caption{$u=(-10,280)$: Sub-optimal solution due to unnecessarily long trajectory, $J=378$.}
    \label{fig:opt_contour_3}
\end{figure}

\begin{figure}[H]
    \centering
    \includegraphics[width=.8\linewidth]{J_sim_67_147}
    \caption{$u=(67,147)$: Optimal solution, $J=186$.}
    \label{fig:opt_contour_4}
\end{figure}

Since there are only two free parameters, the north and east coordinates of $u$, an approximate optimum could be found by performing a grid search over different values of these parameters. However, this solution would depend on the discretization interval of the grid 
and searching over a grid with sufficiently fine resolution is computationally expensive.
A more efficient method is to use \textit{derivative-free} optimization methods, as presented in \cite{derivative_free_opt}. 
In those methods the optimization problem is formulated as 
\begin{subequations}
    \label{eq:derivative_free_opt}
    \begin{alignat}{3}
    &\min_{\xi\in\reals^n}        &\qquad& F: \xi \rightarrow \reals & \\
    &\text{subject to} & & \xi\in\Omega\subseteq \reals^n&\\
    \end{alignat}
\end{subequations}
where no other information, such as the derivatives of $F$, is available.
One class of derivative-free methods called \ac{mads} was introduced in \cite{mads}. This method is based on creating an increasingly fine grid around the currently optimal 
solution on which the objective function is evaluated. In \cite{mads} this method is shown to successfully converge to the global optimum of various non-convex optimization problems using the derivative-free optimization formulation.

\subsection{Robustness during wind variations}
The requirement to generate a set of inputs for each possible wind speed limits 
the practical applicability of the method.
A more useful approach is to generate input sets which handle wind speeds 
$\windspd\in[W_{\text{min}},W_{\text{max}}]$. This problem can be formulated as finding an input $u$ which is feasible for both 
$W_{\text{min}}=(1-\delta_W)\Tilde{\windspd}$ and $W_{\text{max}}=(1+\delta_W)\Tilde{\windspd}$ for some $\delta_W<1$ and $\Tilde{\windspd}=(W_{\text{max}}-W_{\text{min}})/2$. 
To find a solution which is feasible in the extreme cases $\windspd=\windspd_{\text{min}}$ and $\windspd=W_{\text{max}}$, the derivative-free optimization problem was formulated as
\begin{subequations}
    \label{eq:max_opt}
    \begin{alignat}{3}
    &\min_{x, u}        &\qquad& F(x, u)=\max(J_{\text{low}}(x, u),J_{\text{high}}(x, u)) & \\
    &\text{subject to} & & (x, u)\in\Omega &\\
    \end{alignat}
\end{subequations}
where $J_{\text{low}}$ is the value of the objective $J$ in \eqref{eq:opt_problem_mp_uav} for $\windspd=W_{\text{min}}$ and $J_{\text{high}}$ is the value of the objective for $\windspd=W_{\text{max}}$. 
The feasible set $\Omega$ is defined as the values of $x$ and $u$ where the constraints in \eqref{eq:opt_problem_mp_uav} hold for all $\windspd\in[W_{\text{min}},W_{\text{max}}]$.

\section{Improvement step}
As mentioned in Section \ref{sec:hybrid-a-star} the initial solution from Hybrid $A^*$ is often improved using numerical optimization. 
However, due to the limitations presented in Section \ref{sec:solve_opt_ctrl} such methods are not available. Therefore, a simpler and practically motivated approach was used.

The initial solution computed by the Hybrid $A^*$ search is henceforth denoted
\begin{equation}
    \mathcal{M}_{\text{init}}=\{\vec{p}_0,\hdots,\vec{p}_n\}    
\end{equation}
which is an ordered sequence of $n$ waypoints $\vec{p}_i$. A sub-sequence of a mission is denoted
\begin{equation}
    \mathcal{M}_{k:l}=\{\vec{p}_k,\hdots,\vec{p}_l\}, \quad 0\leq k<l\leq n
\end{equation}
A \textit{reduced set} of waypoints is defined as 
\begin{equation}
    \mathcal{M}_{k,l}=\{\vec{p}_k,\vec{p}_l\}
\end{equation}
\ie\ the first and last waypoint of a sub-sequence $\mathcal{M}_{k:l}$. By simulating the closed-loop system using $\mathcal{M}_{\text{init}}$ the inital \ac{cog} and cross-track error $(\cog,d)_i$ at each waypoint can be found. 
Since \eqref{eq:closed_loop} minimizes the cross-track error in each timestep, the following relation always holds:
\begin{equation}
    L(\mathcal{M}_{k:l})\geq L(\mathcal{M}_{k,l})
\end{equation}
where $L(\cdot)$ denotes the length of the trajectory produced by simulating \eqref{eq:closed_loop} with a given waypoint sequence. If the same \ac{cog} and cross-track error 
is achieved and there are no collissions with obstacles while using $\mathcal{M}_{k,l}$ the intermediate waypoints of $\mathcal{M}_{k:l}$ can be eliminated. This method is outlined in Algorithm \eqref{alg:imp} where 
the function $\text{SIMULATE}(\mathcal{M},\xobst)$ returns the \ac{cog} and cross-track error achieved by simulating $\mathcal{M}$ and if there were any 
collissions with $\xobst$. The result of applying the improvement step to a Hybrid $A^*$ solution is illustrated in Figure~\ref{fig:imp}.

\begin{algorithm}
    \begin{algorithmic}
        \Require Initial mission $\mathcal{M}_{\text{init}}$ and corresponding \ac{cog} and cross-track errors $\{(\cog,d)_i\}$
        \State $\mathcal{M}_{\text{imp}}\gets \{\vec{p}_0\}$
        \State $i\gets 0$
        \While{$i\leq n$}
            \State $j\gets i+1$
            \State $(\vec{p}_{\text{best}},i_{\text{best}})\gets(\vec{p}_j,j)$
            \While{$j\leq n$}
                \State $\cog,d,\text{has\_collided}\gets\text{SIMULATE}(\mathcal{M}_{i,j}, \xobst)$
                \If{\textbf{not} has\_collided \textbf{and}$|\cog-\psi_{\text{cog},j}|\leq\Delta\psi_{\text{min}}$ \textbf{and} $|d-d_j|\leq d_{\text{min}}$}
                    \If{j==n}
                        \State $\mathcal{M}_{\text{imp}}\gets \mathcal{M}_{\text{imp}} \bigcup \{\vec{p}_j\}$
                        \State \textbf{return} $\mathcal{M}_{\text{imp}}$
                    \EndIf
                    \State $(\vec{p}_{\text{best}},i_{\text{best}})\gets(\vec{p}_j,j)$
                \EndIf
            \EndWhile
            \State $\mathcal{M}_{\text{imp}}\gets \mathcal{M}_{\text{imp}} \bigcup \{\vec{p}_j\}$
            \State $i\gets i_{\text{best}}$
        \EndWhile 
    \end{algorithmic}
    \caption{Solution improvement by waypoint elimination}
    \label{alg:imp}
\end{algorithm}

\begin{figure}
    \begin{center}
        \includegraphics[width=\linewidth]{fig/sol_improved}
        \caption{Trajectory length reduction by eliminating waypoints. $x_0=(0,0,0\degree)$, $x_g=(-615, 245, 180\degree)$. By eliminating the intermediate waypoints in the sub-mission $\mathcal{M}_{4:8}$ the same goal state is reached but the trajectory length is reduced.}
        \label{fig:imp}
    \end{center}
\end{figure}

\section{Heuristic function}
As discussed in Section \ref{sec:a-star}, the choice of heuristic function is 
crucial in achieving good performance of the planner. The goal of the heuristic function is to estimate the 
length of the shortest path relative to the air from an initial state $x_0$ to a final state $x_g$.

\subsection{Cost estimation for straight line-segments}\label{sec:straight_path_heuristic}
Assuming that the heading $\psi$ has converged to $\wca$, the speed of the \ac{uav} 
along a straight line-segment in the inertial frame is given by 
\begin{equation}
    V_{\parallel}=\cos\psi_s(\airspd\cos\wca+\windspd\cos\winddir) + \sin\psi_s(\airspd\sin\wca+\windspd\sin\winddir)
\end{equation}
where $\psi_s$ is the direction defined by the line. This means that the time it takes for the \ac{uav} to travel along the line is equal to 
\begin{equation}
    t=\frac{\|\vec{p}_{i+1}-\vec{p}_i\|}{V_\parallel}
\end{equation}
where $\vec{p}_{i}$ and $\vec{p}_{i+1}$ are the start and end waypoints of the line. Thus, the distance 
travelled relative to the air is equal to 
\begin{equation}
    s_a=V_at=\frac{\airspd}{V_\parallel}\|\vec{p}_{i+1}-\vec{p}_{i}\|
\end{equation}
and $s_a$ provides a good heuristic estimate for traveling along a straight line-segment in wind assuming that $\psi_0=\psi_g=\wca$.
This also implies that the Euclidean distance $\|\vec{p}_{i+1}-\vec{p}_{i}\|$ is not an admissible heuristic if 
$\airspd/V_\parallel<1$.

\subsection{Cost estimation for arbitrary initial and final heading}
Estimating the cost for traveling between states with arbitrary $\psi_0$ and $\psi_g$ is a more challenging problem than straight line-segments.
Methods to calculate such time-optimal paths in the presence of wind are given in both \cite{optimal_path_target} and \cite{optimal_path_trochoidal}, 
but since there is no general analytical solution these methods rely on numerical root-finding techniques.
Solving for roots numerically every time an heuristic estimate is needed was deemed infeasible due to the high computational cost.

When the heuristic cannot be calculated in real-time, an option is to use a \ac{hlut} as discussed in Section \ref{sec:hlut}. By using the generated 
inputs $\inputs$ when calculating costs stored in the \ac{hlut}, these directly correspond to the true cost-to-go. However, a drawback of 
using a \ac{hlut} is that the wind speed $\windspd$ affects the cost, and thus different values of $\windspd$ require different \acp{hlut}. 

To estimate the cost of queries not stored in the \ac{hlut}, these queries can be projected as shown in Figure \ref{fig:hlut_proj}. The total heuristic value can then be estimated as 
\begin{equation}
    \tilde{h}(x, x_g) = h_{\ac{hlut}}(x, x_p) + h_s(x_p, x_g)
\end{equation}
where $h_s(x, \tilde{x})$ is the estimated cost for a straight line-segment.
\begin{figure}
    \begin{center}
        \begin{tikzpicture}
            \draw (-2, -2) rectangle (2, 2);

            \draw[my_v] (0,0) -- node[at end, below]{$y_E$} (1,0);
            \draw[my_v] (0,0) -- node[at end, left]{$x_N$} (0,1);

            \node[point] at (0,0){};
            \node[below] at (0,0){$x$};

            \node[point] at (2,1){};
            \node[above] at (2,1){$x_p$};
            \draw (0,0) -- node[midway, below, anchor=north west]{$h_{\ac{hlut}}(x, x_p)$} (2,1);

            \node[point] at (4,2){};
            \node[above] at (4,2){$x_g$};
            \draw[dashed] (2,1) -- node[midway,below, anchor=north west]{$h_s(x_p,x_g)$} (4,2);

            \node[anchor=west] at (-1.8,-1.75){\ac{hlut} available};
        \end{tikzpicture}
    \end{center}
    \caption{Projection of queries on \ac{hlut}}
    \label{fig:hlut_proj}
\end{figure}

\subsection{Wind variation effects on the heuristic}
If the actual wind speed $\tilde{\windspd}$ is different from the wind speed $\windspd$ used during planning, this 
might affect the admissibility of the heuristic. 
To study this effect, consider traveling along a straight path segment of length $\Delta s=\|x-\tilde{x}\|$ under the assumptions in Section \ref{sec:straight_path_heuristic}. 
An admissible heuristic is then 
\begin{equation}
    \tilde{h}(x, \tilde{x})=\frac{\airspd}{V}\Delta s
\end{equation}
where $V$ is the velocity in the inertial frame. Wind has the largest effect on $V$ when traveling in direct tailwind or headwind, and in those cases $V=\airspd\pm \tilde{\windspd}$. The 
heuristic function $h(x, \tilde{x})$ used during planning is the same but with $V=\airspd\pm \windspd$. The ratio between the admissible and actual heuristics becomes
\begin{equation}\label{eq:wind_heuristic_eps}
    \epsilon = \frac{h}{\tilde{h}} = \frac{\airspd \pm \tilde{\windspd}}{\airspd \pm \windspd}
\end{equation}
where the signs in the numerator and denominator are always equal. As mentioned in Section \ref{sec:sub_optimal} the heuristic is not admissible if 
$\epsilon>1$ which is the case if $\tilde{\windspd}>\windspd$ when traveling in tailwind, or $\tilde{\windspd}<\windspd$ when traveling in headwind.
In these cases, using this heuristic estimate is analouge to using an inflated heuristic with inflation factor $\epsilon$. Moreover, the effects of using an incorrect wind estimate will be more significant if the 
magnitude of $\tilde{\windspd}$ is close to that of $\airspd$.
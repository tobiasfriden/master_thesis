\section{Fixed-wing dynamics}
This first section will give the reader a quick overview of some common definitions and terminology. 
A general purpose 6-DOF model of a fixed-wing UAV will also be derived. 
\subsection{General definitions and terminology}
We begin with establishing some common definitions and
 terminology which will be used throughout this thesis. These definitions are 
 used in many other works related to fixed-wing aircraft, such as \cite{uav_dynamics_wind}, \cite{uav_dynamics_modeling}, 
\cite{fighter_aircraft_mpc}. 

\subsubsection{Coordinate reference frames}
Three different coordinate reference frames are used, an inertial frame, a body frame and a wind reference frame.
This is convenient since some sensors, like GPS report their values in the inertial frame
 while others, like IMU sensors report values in the body frame. Other advantages of the different references
will be made apparent when formulating the dynamic equations.

\begin{definition}[Inertial frame]
    The inertial frame, denoted with subscript $I$ is fixed relative to the earth.
    A position vector in the intertial frame is defined in the NED order as
    \begin{equation}
        \vec{p}_I = (p_N, p_E, -p_H)
    \end{equation}
    where $p_N$ points in the north direction, $p_E$ points east and $p_H$ points down towards the earth,
    in order to form a right hand positive coordinate system.
\end{definition}

\begin{definition}[Body frame]
    The body frame, denoted with subscript $B$ is fixed in the UAV center of gravity.
    A position vector in the body frame is defined as
    \begin{equation}
        \vec{p}_B = (x, y, z)
    \end{equation}
    where $x$ points forward through the UAV, $y$ points to the right and $z$ points down as shown in Figure TODO
\end{definition}

\begin{definition}[Wind reference frame]
    The wind reference frame, denoted with subscript $W$ is related to the current direction of motion
    through the air.
    A position vector in the wind reference frame is defined as
    \begin{equation}
        \vec{p}_W = (x_w, y_w, z_w)
    \end{equation}
    where $x_w$ points in the same direction as the current velocity vector $\vec{v}_I$, 
    $y_w$ points to the right of $x_w$ and $z$ points down relative $x_w$ and $y_w$.
\end{definition}

\subsubsection{Attitude representation}
The attitude of the UAV is represented by the \textit{Euler angles}. 

\begin{definition}[Euler angles]
The Euler angle vector is defined as
\begin{equation}
    \vec{\Phi}=(\phi, \theta, \psi)
\end{equation}
where the \textit{roll angle} $\phi$ is rotation around the north inertial axis, 
the \textit{pitch angle} $\theta$ is rotation around the east inertial axis and
the \textit{yaw angle} $\psi$ is rotation around the downwards inertial axis.
\end{definition}

The relationship between coordinates in the body frame and inertial frame is given
 by the rotation matrix

\begin{equation}\label{eq:r_i_b}
\mathcal{R}^I_B = \mathcal{R}^x_\phi\mathcal{R}^y_\theta\mathcal{R}^z_\psi\\
=
\begin{bmatrix}
    1 & 0 & 0 \\
    0 & \cos\phi & \sin\phi \\
    0 & -\sin\phi & \cos\phi
\end{bmatrix}
\begin{bmatrix}
    \cos\theta & 0 & -\sin\theta \\
    0 & 1 & 0 \\
    \sin\theta & 0 & \cos\theta
\end{bmatrix}      
\begin{bmatrix}
    \cos\psi & \sin\psi & 0 \\
    -\sin\psi & \cos\psi & 0 \\
    0 & 0 & 1
\end{bmatrix}
\end{equation}  
This attitude representation is not defined for $\theta=\pm\pi/2$. However, such attitudes
 were deemed very unlikely in this work as the main focus is on level flight scenarios.

\subsubsection{Aerodynamic angles}
The \textit{aerodynamic angles} relate the wind reference frame to the other coordinate frames.

\begin{definition}[Aerodynamic angles]
We first define the body frame velocity vector as 
\begin{equation}
    \vec{v}_B=(u, v, w)
\end{equation}
The \textit{angle of attack} $\alpha$ and \textit{side slip} $\beta$ are then defined as
\begin{equation}
    \alpha=\arctan\frac{w}{u}
\end{equation}
\begin{equation}
    \beta=\arcsin\frac{v}{V_a}
\end{equation}
where $V_a=\sqrt{u^2+v^2+w^2}$
\end{definition}


\subsection{Dynamic equations}
In the following section the dynamic equations of a general fixed-wing aircraft will be derived. The 
equations are based on the following common state representation:
\begin{equation}
    \vec{x}=
    \begin{bmatrix}
        \vec{p}_I & \vec{v}_B & \vec{\Phi} & \vec{\omega}
    \end{bmatrix}
\end{equation}
where $\vec{\omega}=(p, q, r)$ are the angular rates in the body reference frame. These can be divided
into translational and rotational dynamics which are presented separately.
\subsubsection{Translational dynamics}
The dynamic equation for the position $\vec{p}_I$ is directly given as
\begin{equation}\label{eq:dynamic_first}
    \dot{\vec{p}}_I=\mathcal{R}^I_B\vec{v}_B
\end{equation}
Furthermore, Newtons second law of motion gives 
\begin{equation}\label{eq:uav_force}
    \vec{F}_{tot}=m\dot{\vec{v}}_B + \vec{\omega}\times m\vec{v}_B 
\end{equation}
where $m$ is the UAV mass and $\vec{F}_{tot}$ is the sum of all forces working on the UAV.
These can be divided into
\begin{equation}
    \vec{F}_{tot}=\vec{F}_g+\vec{F}_{aero}+\vec{F}_{thr}
\end{equation}
where $\vec{F}_g$ is the gravitational force, $\vec{F}_{aero}$ is the aerodynamic force and
$\vec{F}_{thr}$ is the thrust force. The thrust force is assumed to only work in the same direction
as the x-axis in the body frame, so that
\begin{equation}
    \vec{F}_{thr, B}=(T, 0, 0)
\end{equation}
\subsubsection{Rotational dynamics}
The dynamic equations for the Euler vector $\vec{\Phi}$ are given by
\begin{equation}
    \begin{bmatrix}
        \dot{\phi} \\
        \dot{\theta} \\
        \dot{\psi}
    \end{bmatrix}
    =
    \begin{bmatrix}
        1 & -\sin\phi\tan\theta & \cos\phi\tan\theta \\
        0 & \cos\phi & \sin\phi \\
        0 & -\sin\phi\sec\theta & \cos\phi\sec\theta
    \end{bmatrix}
    \begin{bmatrix}
        p \\
        q \\
        r
    \end{bmatrix}
\end{equation}
By again using Newtons second law for the moment we derive
\begin{equation}
    \vec{M}=\vec{I}\dot{\vec{\omega}} + \vec{\omega}\times\vec{I}\vec{\omega}
\end{equation}
The moment $\vec{M}$ is assumed to mainly come from aerodynamics.
\subsubsection{Aerodynamics}
The aerodynamic forces and moments generated require some further analysis. The aerodynamic
force $\vec{F}_{aero}$ can be written in the wind reference frame as
\begin{equation}
    \vec{F}_{aero}=
    \begin{bmatrix}
        -D \\
        Y \\
        -L
    \end{bmatrix}=
    \begin{bmatrix}
        \bar{q}SC_D \\
        \bar{q}SC_Y \\
        \bar{q}SC_L
    \end{bmatrix}
\end{equation}
where $\bar{q}=\frac{1}{2}\rho(h)V^2$ is the free-stream dynamic pressure, $S$ is the 
wing surface area and $C_D$, $C_Y$ and $C_L$ are dimensionless constants.
$D$, $Y$ and $L$ are called the \textit{drag}, \textit{sideforce} and \textit{lift} respectively. 
The corresponding coefficients are dependent on a number of variables, such as the aerodynamic angles
and airframe dependent parameters \cite{fighter_aircraft_mpc}.\\

The aerodynamic moments are usually defined in the body reference frame as
\begin{equation}\label{eq:dynamic_last}
    \vec{M}_{aero}=
    \begin{bmatrix}
        M_x \\
        M_y \\
        M_z
    \end{bmatrix}=
    \begin{bmatrix}
        \bar{q}SbC_l \\
        \bar{q}SbC_m \\
        \bar{q}SbC_n
    \end{bmatrix}
\end{equation}
where $b$ is the wing-span of the aircraft and $C_l$, $C_m$ and $C_n$ are dimensionless 
constants mainly dependent on the aerodynamic angles \cite{uav_dynamics_wind}. \\

\subsubsection{Wind effects}
The dynamic equations derived in \eqref{eq:dynamic_first}-\eqref{eq:dynamic_last} are under
the assumption that the air through which the UAV is travelling is at rest relative to the earth.
During real flights, wind will cause the air to move which has non-negligible effects on the UAV.
The wind vector is often defined in the inertial frame as
\begin{equation}
    \vec{w}=(w_N, w_E, w_H)
\end{equation}
$\vec{w}$ can be decomposed to
\begin{equation}
    \vec{w}=\vec{w}_{m}+\vec{w}_s
\end{equation}
where $\vec{w}_m$ is the mean wind and $\vec{w}_s$ is some stochastic process \cite{spline_trajectory}. In this work,
we only consider the mean wind.
The mean wind magnitude $W=|\vec{w}|$ varies with altitude as
\begin{equation}
    W(h)=W_{10}(\frac{h}{h_{10}})^a
\end{equation}
where $W_{10}$ is the wind measured at $h=10$ meters and $a$ is the Hellman exponent, which depends on
the shape and coastal location of the underlying terrain \cite{uav_dynamics_wind}. The wind can be 
incorporated in the dynamic equations of motion by extending \eqref{eq:dynamic_first} to
\begin{equation}
    \dot{\vec{p}}=\mathcal{R}^I_B\vec{v}_B+\vec{w}
\end{equation}
Also, we need to extend \eqref{eq:uav_force} by replacing $\vec{v}_B$ with
\begin{equation}
    \vec{v}_g = \vec{v}_B + \vec{w}_B
\end{equation} 
where $\vec{v}_g$ is the body-fixed velocity relative the ground and $\vec{w}_B$ is the
wind vector in the body frame.

\subsubsection{Complete dynamics model}
The complete dynamic model is given by combining the above equations. It will not be 
written out here, but is given in e. g. \cite{uav_dynamics_wind}. The dynamics are clearly nonlinear and quite complex. Extensive modeling and experimentation, such as described in \cite{aerodynamic_parameters}
is needed to identify the different parameters which describe the behaviour of a specific airframe.

It is often interesting to study the \textit{longitudal} and \textit{lateral} dynamics
separately, where \textit{longitudal} denotes movement in the vertical plane and
\textit{lateral} denotes movement out of the vertical plane. The decoupled state
vectors are given by
\begin{equation}
    \vec{x}_{lng}=(p_N, p_H, u, w, \theta, q)
\end{equation}
and
\begin{equation}
    \vec{x}_{lat}=(p_E, v, \phi, \psi, p, r)
\end{equation}
By introducing the polar inertial components $(V, \gamma)$ and writing the dynamic equations
in the wind reference frame the longitudal dynamics become
\begin{align}
    \dot{V}&=\frac{1}{m}[T\cos\alpha-D-mg\sin\gamma] \\
    \dot{\gamma}&=\frac{1}{mV}[T\sin\alpha + L - mg\cos\gamma] \\
    \dot{q}&=\frac{M_y}{I_{yy}} \\
    \dot{\alpha}&=\dot{\theta}-\dot{\gamma}=q-\frac{1}{mV}[T\sin\alpha + L - mg\cos\gamma]
\end{align}
Assuming level flight in the longitudal direction, i. e. $u$, $V$ and $p_H$ are constant, the lateral dynamics can be formulated as
\begin{align}
    \dot{p}_E &= u\sin\psi + v\cos\phi\cos\psi \\
    \dot{\psi} &= r\cos\phi \\
    \dot{\phi} &= p \\
    \dot{v} &= \frac{Y_B}{m} + g\sin\phi - ru \\
    \dot{p} &= \frac{I_{zz}M_x + I_{xz}M_z}{I_{xx}I_{zz} - I^2_{xz}} \\
    \dot{r} &= \frac{I_{xz}M_x + I_{xz}M_z}{I_{xx}I_{zz} - I^2_{xz}} \\
\end{align}
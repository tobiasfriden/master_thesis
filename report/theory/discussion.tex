\chapter{Conclusions}\label{cha:discussion}
\section{Results}
In this work, a novel method to automatically generate landing sequences for fixed-wing \acp{uav} is proposed. 
The method automatically handles many of the challenging aspects when specifying such a sequence manually, \eg\ determining the current 
direction and speed of the wind. 

The results in Chapter \ref{cha:results}, from both simulation and real flight tests, indicate that the method successfully generates feasible landing 
sequences in different wind conditions. The error in distance to the landing point is very small relative to the total length of the landing sequence.
The relative magnitude of the error in entry altitude is larger, but the \ac{uav} still manages to enter $\landing$ above the specified $h_{\text{safe}}$ in most cases. 
The error between calculated entry altitude and actual entry altitude seems quite constant, at least in the simulated evaluations summarized in Table \ref{tab:opt_land_param}. 
This implies that the error could be mitigated by estimating this offset and adding it to the desired $h_{\text{safe}}$. The error could also be mitigated by scaling the second term in the objective of 
Equation \ref{eq:opt_problem_land} with some constant $\lambda_h>1$. The landing sequence generation is also quite fast, and a solution is found in well below 1 second in most cases.

The proposed framework is quite general and could be implemented on any fixed-wing \ac{uav} which uses the trajectory controller described in Section \ref{sec:traj_controller}. 
It would also be easy to extend it to support another controller, as the only requirement to create the input set $\inputs$ is that the closed-loop model of the system is written on the same form as 
Equation \ref{eq:closed_loop}.

\section{Limitations}
Constraining the control reference to consist of straight line-segments or waypoints limits the system from more complex trajectories as the ones used in \cite{emergency_landing}. However, 
most popular autopilots such as Ardupilot use this formulation \cite{arduplane}. 

Sampling methods in general contain inherent limitations, as the 
quality of the solution depends on the resolution of the sampling. In many cases, such as when generating a \ac{hlut} there is also a tradeoff between resolution and storage capacity. Increasing the resolution 
with a factor of \eg\ 2 increases the size of the \ac{hlut} with a factor of 4. However, calculating the solutions analytically is often not a feasible alternative, especially if realtime performance is desired, as is 
often the case in robotics applications.

A large limitation in the landing area definition in Chapter \ref{cha:landing} is the assumption that the terrain elevation is constant inside $\landing$. In 
most real-world cases, such as landing in a slope, the terrain elevation varies. Including this factor in the landing sequence generation would enable landings in a much wider class of scenarios.

\section{Future work}
The landing sequence framework depends on many different parameters, both airframe-specific such as $\dot{\psi}_{\text{max}}$ and general such as discretization step-sizes and which wind speeds are used for input generation.
Since the goal of this thesis was mainly to evaluate the feasibility of the proposed method, most of those parameters were set to "good-enough" values which proved to work but are not necessarily optimal. 
A future work consists of tuning these parameters for the currently used \ac{uav} platform, which would require a number of real-world flight experiments in different wind conditions.
It would also be interesting to study methods to efficiently and automatically estimate optimal values of these parameters, especially those specific to the airframe.

As mentioned in the previous section, an important future work is to include terrain elevation in the landing sequence generation. Another important area is to 
study how an additional system mounted on the \ac{uav} could automatically detect suitable landing areas, \eg\ using vision sensors and an elevation and obstacle database. 
This would be a step towards truly autonomous fixed-wing \acp{uav}, as the system would be able to perform a safe landing without any pilot input. It could also be used to perform 
emergency landings if the command and control link to the pilot is lost.
\chapter{Conclusions}\label{cha:discussion}
\section{Results}
In this work, a novel method to automatically generate landing sequences for fixed-wing \acp{uav} is proposed. 
The method automatically handles many of the challenging aspects when specifying such a sequence manually, \eg\ taking the current 
direction and speed of the wind into account.

The method consists of two main components, the \textit{landing sequence calculation} and \textit{motion planner}. Both these components mainly rely on optimization-based methods. 
To calculate the parameters determining the landing sequence, \ie\ when the \ac{uav} descends to the ground, a nonlinear optimization problem is 
formulated and solved numerically. The motion planner uses a set of precomputed waypoints together with sampling-based planning techniques to determine a plan which is feasible given the 
\ac{uav} model and current wind conditions. To create the set of waypoints, an optimization problem is solved using derivative-free optimization.

The proposed method is quite general and could be implemented on any fixed-wing \ac{uav} which uses the trajectory controller described in Section \ref{sec:traj_controller}. 
It would also be easy to extend it to support another controller, as the only requirement to create the input set $\inputs$ is that the closed-loop model of the system is written on the same form as 
Equation \eqref{eq:closed_loop}.

\section{Limitations of the method}
Constraining the control reference to consist of waypoints, \ie\ straight line-seg\-ments, limits the system from using more complex trajectories like the ones used in \cite{emergency_landing}. It also introduces some issues mentioned in Chapter \ref{cha:results}. 
However, most popular autopilots, such as Ardupilot, use this formulation \cite{arduplane}. 

Sampling methods contain inherent limitations, such as the 
quality of the solution depending on the sampling resolution. In many cases, such as when generating a \ac{hlut}, there is also a tradeoff between resolution and storage capacity. In the case of a 2-dimensional \ac{hlut} as in this thesis, the \ac{hlut} size scales quadratically with the sampling resolution used. 
However, calculating analytical solutions in real-time is often infeasible due to high computational costs.

A large limitation of the landing area definition in Chapter \ref{cha:landing} is the assumption that the terrain elevation is constant inside the landing area $\landing$. In 
most real-world cases, such as landing in a slope, the terrain elevation varies. Including this factor in the landing sequence generation would enable landings in a much wider class of scenarios.

The assumption that the wind is constant simplifies many parts of calculating the landing sequence, but such an approximation also introduces limitations in the performance of the method. As can be seen in Figure \ref{fig:real_wind}, the windspeed $\windspd$ is quite constant while the \ac{uav} is flying at a constant altitude. However, there is a clear altitude dependency in the 
wind-speed estimate, which is visible both during the takeoff and landing portions of the flight. It is therefore likely that including this altitude dependency, especially in the calculation of the final descent parameters $\vec{p}_a$ and $\vec{p}_l$ could increase the precision of the landing.

\section{Future work}
The landing sequence method depends on many different parameters, both air\-frame-specific such as $\dot{\psi}_{\text{max}}$ and general such as discretization step-sizes and wind speeds used for input generation.
The goal of this thesis was mainly to evaluate the feasibility of the proposed method. Hence, most of those parameters were set to "good-enough" values which proved to be feasible but are not necessarily optimal. 
A possible future work consists of tuning these parameters for the currently used \ac{uav} platform, which would require a number of real-world flight experiments in different wind conditions.
It would also be interesting to study methods to efficiently and automatically estimate optimal values of these parameters, especially those specific to the airframe.

As mentioned, an important future work is to include terrain elevation in the landing sequence generation. Another important area is to 
study how an additional system mounted on the \ac{uav} could automatically detect suitable landing areas, \eg\ using vision sensors and an elevation and obstacle database. 
This would be a step towards truly autonomous fixed-wing \acp{uav}, as the system would be able to perform a safe landing without any pilot input. It could also be used to perform 
emergency landings if the command and control link to the pilot is lost.